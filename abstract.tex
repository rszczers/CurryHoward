\paragraph{Streszczenie}
Celem tej pracy jest wyłożenie minimalnych podstaw teoretycznych
koniecznych do zrozumienia technik programowania funkcyjnego
z perspektywy matematycznej. Na gruncie rachunku \(\lambda\) bez typów,
poza składnią i semantyką oraz podstawowymi własnościami tego systemu,
omówione zostaną strategie obliczania wyrażeń oraz algebraiczne
typy danych. Omawiamy rozszerzenie rachunku \(\lambda\)  najpierw o typy proste,
a następnie o typy parametryczne celem wyjasnienia procesu kompilacji
programów w języku Haskell do języka pośredniego opartego na 
systemie Girarda/Reynoldsa.

\paragraph{Abstract}
The aim of this thesis is to provide minimal theoretical foundations
necessary to understand functional programming techniques from the
mathematical perspective. Starting form untyped \(\lambda\)-calculus,
besides its syntax, semantics and elementary proporties, we discuss
evaluation strategies and algebraic data types. We extend \(\lambda\)-calculus
with simple types and then with parametric types in order to demonstrate
the process of compilation Haskell programs to its intermediate language,
which is based on Girard/Reynolds system.
