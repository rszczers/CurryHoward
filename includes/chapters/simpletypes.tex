\section{Typy proste w stylu Churcha}

System \(\lambda_{\to}\) w stylu Churcha to zbiór typów \(\mathrm{T}\),
zbiór pseudotermów \(\Lambda_{\mathrm{T}}\), rodzina otoczeń typowych,
relacja \(\beta\)-kontrakcji \(\to_{\beta}\) i relacja przypisania typu \(\vdash\).

\subsection{Język}

\begin{definicja}$ $\newline
\begin{itemize}
\item
  \emph{Typami prostymi} \(\mathrm{T}\) nazywamy zbiór \(\Phi_{\to}\) wszystkich formuł języka logiki NJ(\(\to\)). 
    Zamiast mówić o zmiennych zdaniowych, będziemy używali określenia \emph{zmienne typowe}.

    Za zmienne podmiotowe dla oznaczeń formuł zdaniowych obieramy późniejsze litery alfabetu greckiego, tj. \(\sigma,\, \tau,\, \rho\,\dots\)
\item
  \emph{Pseudo-pretermami} nazywamy język \(\Lambda_{\mathrm{T}}\) generowany przez
gramatykę 
\[
  \Lambda^{-}_{\mathrm{T}} := \ \mathrm{V}\ | \ \left (\lambda V^{\mathrm{T}} . \Lambda^{-}_{\mathrm{T}}\right) \ | \ \left (\Lambda^{-}_{\mathrm{T}}\Lambda^{-}_{\mathrm{T}}\right)
\]
    gdzie V to przeliczalny zbiór \(\lambda\)-zmiennych \(x, y, \dots\)
    
    W języku podmiotowym będziemy używali późniejszych liter alfabetu łacińskiego pisanych kursywą (\(M,\, N,\, O,\, \dots\)) oznaczając pseudotermy.
  \item \emph{Otoczeniem typowym} nazywamy skończoną funkcję częściową \(\Gamma:\:\mathrm{V}\to\mathrm{T}\) przeprowadzającą zbiór \(\lambda\)-zmiennych w zbiór typów prostych. Nadużywając notacji piszemy
    \begin{itemize}
      \item \(\Gamma=\{x_1\,:\,\tau_1,\,\dots,\,x_n\,:\,\tau_n\}\)
      \item \(\mathrm{dom}(\Gamma) = \left\{x\in \mathrm{V}\,|\:\exists\tau.\,(x\,:\,\tau)\in\Gamma\right\}\)
      \item \(\mathrm{rg}(\Gamma)=\left\{\tau\in\Phi_{\to}\,|\,\exists x.\,(x\,:\,\tau)\in\Gamma\right\}\)
    \end{itemize}

  \item Dla pseudotermu \(M\) następująco określamy zbiór \emph{termów wolnych} \(\mathrm{FV}\):
    \begin{align*}
      \mathrm{FV}(x) &= \{x\}\\
      \mathrm{FV}(\lambda x^\sigma .\, P)  &= \mathrm{FV}(P)\setminus\{x\}\\
      \mathrm{FV}(P Q) &= \mathrm{FV}(P)\cup\mathrm{FV}(Q)
    \end{align*}

  \item \emph{Podstawieniem} \([x/N]\) pseudotermu \(N\) za \(\lambda\)-zmienną \(x\) w \(M\) nazwamy zdefiniowane następująco przekształcenie:
  \begin{align*}
    x[x/N] &= N,\\
    y[x/N] &= y,\ &\text{o ile}\ x\neq y,\\
    (PQ)[x/N] &= P[x/N]\,Q[x/N],\\
    (\lambda y^\sigma.\, P)[x/N] &= \lambda y^\sigma .\,P[x/N],\ &\text{gdzie}\ x\neq y\ \text{i}\ y\not\in \mathrm{FV}(N).\\
  \end{align*}
    Jesli \(\mathrm{FV}(M)=\emptyset\), to pseudoterm \(M\) nazywamy \emph{zamkniętym}. 
%  \item Podstawieniem pseudotermu \(N\) za \(\lambda\)-zmienną \(x\) w pseudotermie \(M\), symbolicznie \(M[x/N]\), nazywamy przekształcenie pseudotermu zadane następującymi warunkami:
%    \begin{enumerate}[label=({\alph*})]
%    \item Podstawienie jest poprawne wtedy i tylko wtedy, gdy żadne wolne wystąpienie \(\lambda\)-zmiennej \(x\) w pseudotermie \(M\) nie występuje w podtermie \(M\) postaci \(\lambda y.\, L\), gdzie \(y\in\mathrm{FV}(N)\).
%    \item
%      \begin{align*}
%        x[x/N] &= N,&\\
%        y[x/N] &= y,\ &\text{o ile}\ x\neq y,&\\
%        (PQ)[x/N] &= P[x/N]\,Q[x/N],&\\
%        (\lambda x.\,P)[x/N] &= \lambda x.\,P,&\\
%        (\lambda y.\,P)[x/N] &= \lambda y.\,P [x/N],\ &\text{o ile}\ x\neq y.
%      \end{align*}
%    \end{enumerate}
    \begin{fakt}$ $\\
      \begin{enumerate}[label=({\alph*})]
        \item Jeśli \(x\not\in\mathrm{FV}(M)\), to \(M[x/N]\) jest poprawnym podstawieniem i \(M[x/N]=M\).
        \item Jeśli \(M[x/N]\) jest poprawnym podstawieniem, to \(y\in\mathrm{FV}(M[x/N]\) wtw, gdy albo \(y\in\mathrm{FV}(M)\)
          i \(x\neq y\), albo \(y\in \mathrm{FV}(N)\) i \(x\in \mathrm{FV}(M)\). 
        \item Podstawienie \(M[x/x]\) jest poprawne i \(M[x/x]=M\).
        \item Jeśli \(M[x/y]\) jest poprawnym podstawieniem, to \(M[x/y]\) ma tę samą długość, co \(M\).
      \end{enumerate}
    \end{fakt}
    \begin{fakt}
      Powiedzmy, że \(M[x/N]\) jest poprawnym podstawieniem i \(N[y/L]\) i \(M[x/N][y/L]\) są poprawnymi podstawieniami, gdzie
      \(x\neq y\). Jeśli \(x\not\in \mathrm{FV}(L)\) lub \(y\not\in\mathrm{FV}(M)\), to \(M[y/L]\) i \( M[y/L]\left[x/N[y/L]\right] \) jest poprawnym podstawieniem oraz
      \[
        M[x/N][y/L]=M[y/L][x/N[y/L]].
      \]
    \end{fakt}

    \begin{fakt}
      Jesli \(M[x/y]\) jest poprawnym postawieniem i \(y\not\in\mathrm{FV}(M)\), to \(M[x/y][y/x]\) jest poprawnym podstawieniem oraz
      \(M[x/y][y/x]=M\).
    \end{fakt}

  \item \emph{\(\alpha\)-konwersją} \(=_\alpha\) nazywamy najmniejszą w sensie mnogościowym relację zwrotną i przechodnią określoną na zbiorze pseudotermów \(\Lambda^{-}_{\mathrm{T}}\) spełniającą poniższe warunki:
    \begin{enumerate}[label=({\alph*})]
      \item Jeśli \(y\not\in \mathrm{FV}(M)\) i \(M[x/y]\) jest poprawnym podstawieniem, to \(\lambda x.\, M  =_\alpha \lambda y.\, M[x/y]\).
      \item Jeśli \(M=_\alpha N\), to dla każdej \(\lambda\)-zmiennej \(x\) mamy \(\lambda x.\, M =_\alpha \lambda x.\,N\).
      \item Jeśli \(M=_\alpha N\), to \(M Z=_\alpha N Z\).
      \item Jeśli \(M=_\alpha N\), to \(ZM =_\alpha ZN\).
    \end{enumerate}
    \begin{fakt}
      Relacja \(=_{\alpha}\) jest symetryczna.
    \end{fakt}
    \begin{fakt}
      \(=_{\alpha}\) jest relacją równoważności.
    \end{fakt}
    \begin{fakt}
      Jeśli \(M=_\alpha N\), to \(\mathrm{FV}(M)=\mathrm{FV}(N)\).
    \end{fakt}
  \item \emph{Pseudotermami} nazywamy zbiór ilorazowy \(\Lambda_\mathrm{T}\) relacji \(\alpha\)-konwersji
  \[
    \Lambda_{\mathrm{T}}=\left\{[M]_\alpha\:|\: M\in\Lambda^{-}_\mathrm{T}\right\}
  \]
\item
    \emph{Sądem} (\emph{asercją}) nazywamy każdą trójkę \((\Gamma ,\,M,\,\sigma )\in\mathcal{P}(\mathrm{V}\times \mathrm{T})\times\Lambda_{\mathrm{T}}\times\mathrm{T}\), gdzie \(\Gamma\) jest otoczeniem typowym i oznaczamy \(\Gamma\vdash M^{\sigma}\).

  Piszemy: \begin{itemize}
    \item \(\varphi_{1},\ \varphi_{2}\vdash\psi\) zamiast \(\{\varphi_{1},\ \varphi_{2}\}\vdash\psi\),
    \item \(\Gamma, x^\varphi\) zamiast \(\Gamma\cup \{x^\varphi\}\), o ile \(x^\varphi\not\in \Gamma\).
    \item \(\Gamma, \Delta\) zamiast \(\{\Gamma\cup \Delta\}\), o ile \(\Gamma\cap\Delta=\emptyset\).
    \item \(\vdash\varphi\) zamiast \(\emptyset\vdash\varphi\).
  \end{itemize}

\item
  Na zbiorze sądów wprowadzamy relacje określające reguły wyprowadzania termów
    \begin{center}
    \begin{tabular}{ cc}
      {\begin{prooftree}
        \Hypo{ \Gamma, x^{\varphi} \vdash M^{\psi} }
        \Infer1[(Abs)]{\Gamma \vdash (\lambda\, x^{\varphi}.\, M)^{\varphi\to\psi}}
      \end{prooftree}},
      &
      {\begin{prooftree}
        \Hypo{\Gamma \vdash M^{\varphi \to \psi}} \Hypo{ \Gamma \vdash N^{\varphi}}
        \Infer2[(App)]{\Gamma \vdash (MN)^{\psi}}
      \end{prooftree}}.
      \end{tabular}
    \end{center}
    oraz wybieramy spośród sądów jeden aksjomat postaci \(\Gamma, x^\tau\vdash x^\tau\ (\mathrm{Var})\).

    Dowód sądu określamy analogicznie jak w logince NJ(\(\to\)).

Mówimy, że \(M\) jest \emph{termem} typu \(\tau\) w otoczeniu \(\Gamma\), jeśli istnieje dowód sądu \(\Gamma \vdash M^{\tau}\) w powyższym systemie dedukcyjnym.
\end{itemize}
\end{definicja}

\subsection{Redukcja}
\begin{definicja}
\begin{itemize}

  \item Relację \(\mathrm{R}\) na zbiorze pseudotermów \(\Lambda_{\mathrm{T}}\) nazywamy \emph{zgodną}, jeśli dla \(M,\,N,\,Z\in\Lambda_{\mathrm{T}}\) spełnia następujące warunki

  \begin{enumerate}[label=({\alph*})]
    \item Jeśli \(M\mathrm{R} N\), to \((\lambda x^\sigma.\,M)\, \mathrm{R}\, (\lambda x^\sigma.\, N)\) dla każdej \(\lambda\)-zmiennej \(x\) dla której istnieje \(\sigma\in \mathrm{T}\).
    \item Jeśli \(M\mathrm{R} N\), to \((MZ)\,\mathrm{R}\, (NZ)\).
    \item Jeśli \(M\mathrm{R} N\), to \((ZM)\,\mathrm{R}\, (ZN)\).
  \end{enumerate}
  \item \emph{Kongruencją} nazywamy zgodną relację równowazności na \(\Lambda_{\mathrm{T}}\).
  \item \emph{Redukcją} nazywamy zgodną, zwrotną i przechodnią relację na \(\Lambda_{\mathrm{T}}\).

\item \(\beta\)-redukcją nazywamy najmniejsza w sensie mnogościowym \emph{zgodną} relację „\(\longrightarrow_{\beta}\)” określoną zbiorze pseudotermów \(\Lambda_{\mathrm{T}}\) za pomocą podstawienia
  \[
    (\lambda x^\sigma.\,P)Q \longrightarrow_{\beta} P[x/Q].
  \]
  
    \emph{\(\beta\)-redeksem} nazywamy wyrażenia postaci \((\lambda x^\sigma.\, M)N\). Rezultatem \(\beta\)-redukcji jest term postaci \(M[x/N]\), który nazywamy \emph{\(\beta\)-reduktem}.

    Mówimy, że \(\lambda\)-term \(M\) jest w \emph{postaci normalnej}, jeśli żadna jego podformuła nie jest \(\beta\)-redeksem. 
    
    \(M\) \emph{ma postać normalną}, jeśli \(M=_{\beta}N\) dla pewnego \(N\), który jest w postaci normalnej.


    \(\longrightarrow^{+}_{\beta}\) jest przechodnim domknięciem relacji \(\longrightarrow_{\beta}\) w zbiorze pseudotermów \(\Lambda_{\mathrm{T}}\).

    \(\longrightarrow^{*}_{\beta}\) jest domknięciem przechodnio-zwrotnim w \(\Lambda_{\mathrm{T}}\) relacji \(\longrightarrow_{\beta}\), a zatem jest \emph{redukcją}. 

    \(=_{\beta}\) jest najmniejszą relację równowazności zawierającą relację \(\longrightarrow_{\beta}\), a zatem \emph{kongruencją}.


  \begin{fakt}
    Jeśli \(\Gamma\vdash M^\sigma\) i \(M\longrightarrow^{*}_{\beta}N\), to
    \(\Gamma\vdash N^\sigma\).
  \end{fakt}
\item \(\eta\)-redukcją nazywamy najmniejszą (w sensie mnogościowym) \emph{zgodną} relację w \(\Lambda_{\mathrm{T}}\) taką, że
  \[
    \lambda x^\sigma.\, Mx\longrightarrow_{\eta} M,
  \]
    o ile \(x\not\in \mathrm{FV}(M)\).

  \begin{fakt}
    Jeśli \(\Gamma\vdash M^\sigma\) i \(M\longrightarrow^{*}_{\eta}N\), to
    \(\Gamma\vdash N^\sigma\).
  \end{fakt}

\end{itemize}
\end{definicja}
\subsection{Normalizacja}

  \begin{itemize}
\item 
  \(\lambda\)-term \(M\) ma własność \emph{normalizacji} (co symbolicznie oznaczamy \(M\in\mathrm{WN_{\beta}}\)) wtw, gdy \emph{istnieje} ciąg \(\beta\)-redukcji rozpoczynający się od \(M\) i kończący się termem w postaci normalnej \(N\). \(\lambda\)-term \(M\) ma własność \emph{silnej normalizacji} (symbolicznie: \(M\in\mathrm{SN_{\beta}}\)), jeśli wszystkie ciągi \(\beta\)-redukcji rozpoczynające się \(M\) są skończone.

      \emph{Strategią redukcji} nazywamy odwzorowanie \(F:\:\Lambda_{\mathrm{T}}\to\Lambda_{\mathrm{T}}\) takie, że \(F(M)=M\), gdy \(M\) jest w postaci normalnej i \(M\to_{\beta}F(M)\) w przeciwnym wypadku. Mówimy, że strategia \(F\) jest \emph{normalizująca}, jeśli dla każdego \(M\in \mathrm{WN_\beta}\) istnieje \(i\in\mathbb{N}\) takie, że \(F^i (M)\) jest w postaci normalnej.

\begin{twierdzenie}
  Każdy \(\lambda\)-term w stylu Churcha ma postać normalną.
\end{twierdzenie}
  \end{itemize}
