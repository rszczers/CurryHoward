\section{Typy proste w stylu Churcha}

\subsection{Język}

\begin{definicja}
\begin{itemize}
\item
\emph{Typami prostymi} nazywamy język \(\mathrm{T}\) generowany przez gramatykę
\begin{align*}
  \mathrm{T} &:= U\ |\ (\mathrm{T}\to \mathrm{T})\\
    \mathrm{U} &:= \mathrm{p}\ |\ \mathrm{U}'
\end{align*}

\item Napisy powstałe z produkcji U nazywamy \emph{zmiennymi typowymi}. 

\begin{konwencja*}$ $\newline 
  \begin{enumerate}
    \item Zamiast \(\mathrm{p'},\, \mathrm{p''},\, \mathrm{p'''},\, \dots\) używamy kolejno liter \(\mathrm{p},\, \mathrm{q},\, \mathrm{r},\, \dots\)
    \item Zmienne metasyntaktyczne oznaczamy późniejszymi literami greckiego alfabetu, tj. \(\varphi,\, \psi,\, \dots\)
    \item \(\rightarrow\) jest łączna w prawo.
    \item Pomijamy najbardziej zewnętrzne nawiasy.
  \end{enumerate}
\end{konwencja*}

\item
  \emph{Pseudotermami} nazywamy język \(\Lambda_{\mathrm{T}}\) generowany przez
gramatykę 
\[
  \Lambda_{\mathrm{T}} := \ \mathrm{V}\ | \ \left (\lambda V^{\mathrm{T}} . \Lambda_{\mathrm{T}}\right) \ | \ \left (\Lambda_{\mathrm{T}}\Lambda_{\mathrm{T}}\right)
\]
    gdzie V to przeliczalny zbiór zmiennych \(x, y, \dots\)
  \item \emph{Otoczeniem} nazywamy funkcję częściową \(\Gamma:\:\mathrm{V}\to\mathrm{T}\) przeprowadzającą zbiór zmiennych zmiennych typowych w zbiór typów prostych.
    
\item
  Wprowadzamy relację \emph{typizacji} \(\vdash \subset \mathrm{C}\times\Lambda_{\mathrm{T}}\times\mathrm{T}\) jako najmniejszą (w sensie mnogościowym) relację spełniającą reguły

    \begin{center}
      \begin{tabular}{ ccc}
      {\begin{prooftree}
        \Hypo{ \Gamma, x^{\varphi} \vdash x^{\varphi} \ (\mathrm{Ax}) }
        \end{prooftree}},
      &
      {\begin{prooftree}
        \Hypo{ \Gamma, x^{\varphi} \vdash M^{\psi} }
        \Infer1[(\(\rightarrow\)I)]{\Gamma \vdash (\lambda\, x^{\varphi}.\, M)^{\varphi\to\psi}}
      \end{prooftree}},
      &
      {\begin{prooftree}
        \Hypo{\Gamma \vdash M^{\varphi \to \psi}} \Hypo{ \Gamma \vdash N^{\varphi}}
        \Infer2[(\(\rightarrow\)E)]{\Gamma \vdash (MN)^{\psi}}
      \end{prooftree}}.
      \end{tabular}
    \end{center}
    gdzie \(C\subset\mathcal{P}\left(\mathrm{V}\times\mathrm{T}\right)\) jest rodziną wszystkich otoczeń.

Mówimy, że \(M\) jest \emph{termem} typu \(\tau\) w otoczeniu \(\Gamma\), jeśli \(\Gamma \vdash M^{\tau}\).
\end{itemize}
  
\end{definicja}

