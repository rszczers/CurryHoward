\section{Typy proste w stylu Churcha}

\subsection{Język}

\begin{definicja}$ $\newline
\begin{itemize}
\item
  \emph{Typami prostymi} \(\mathrm{T}\) nazywamy zbiór \(\Phi_{\to}\) wszystkich formuł języka logiki NJ(\(\to\)). 
    Zamiast mówić o zmiennych zdaniowych, będziemy używali określenia \emph{zmienne typowe}.

\begin{konwencja*}$ $\newline 
  \begin{enumerate}
    \item Zamiast \(\mathrm{p'},\, \mathrm{p''},\, \mathrm{p'''},\, \dots\) używamy kolejno liter \(\mathrm{p},\, \mathrm{q},\, \mathrm{r},\, \dots\)
    \item Zmienne metasyntaktyczne oznaczamy późniejszymi literami greckiego alfabetu, tj. \(\sigma,\, \tau,\, \rho\,\dots\)
    \item ,,\(\rightarrow\)'' jest łączna w prawo.
    \item Pomijamy najbardziej zewnętrzne nawiasy.
  \end{enumerate}
\end{konwencja*}

\item
  \emph{Pseudotermami} nazywamy język \(\Lambda_{\mathrm{T}}\) generowany przez
gramatykę 
\[
  \Lambda_{\mathrm{T}} := \ \mathrm{V}\ | \ \left (\lambda V^{\mathrm{T}} . \Lambda_{\mathrm{T}}\right) \ | \ \left (\Lambda_{\mathrm{T}}\Lambda_{\mathrm{T}}\right)
\]
    gdzie V to przeliczalny zbiór \(\lambda\)-zmiennych \(x, y, \dots\)
    
    W języku podmiotowym będziemy używali późniejszych liter alfabetu łacińskiego (M, N, O, …) oznaczając pseudotermy.
  \item \emph{Otoczeniem typowym} nazywamy skończoną funkcję częściową \(\Gamma:\:\mathrm{V}\to\mathrm{T}\) przeprowadzającą zbiór \(\lambda\)-zmiennych w zbiór typów prostych. Nadużywając notacji piszemy
    \begin{itemize}
      \item \(\Gamma=\{x_1\,:\,\tau_1,\,\dots,\,x_n\,:\,\tau_n\}\)
      \item \(\mathrm{dom}(\Gamma) = \left\{x\in \mathrm{V}\,|\:\exists\tau.\,(x\,:\,\tau)\in\Gamma\right\}\)
      \item \(\mathrm{rg}(\Gamma)=\left\{\tau\in\Phi_{\to}\,|\,\exists x.\,(x\,:\,\tau)\in\Gamma\right\}\)
    \end{itemize}

\item
    \emph{Sądem} (\emph{asercją}) nazywamy każdą trójkę \((\Gamma ,\,M,\,\sigma )\in\mathcal{P}(\mathrm{V}\times \mathrm{T})\times\Lambda_{\mathrm{T}}\times\mathrm{T}\), gdzie \(\Gamma\) jest otoczeniem typowym i oznaczamy \(\Gamma\vdash M^{\sigma}\).

  Piszemy: \begin{itemize}
    \item \(\varphi_{1},\ \varphi_{2}\vdash\psi\) zamiast \(\{\varphi_{1},\ \varphi_{2}\}\vdash\psi\),
    \item \(\Gamma, x^\varphi\) zamiast \(\Gamma\cup \{x^\varphi\}\), o ile \(x^\varphi\not\in \Gamma\).
    \item \(\Gamma, \Delta\) zamiast \(\{\Gamma\cup \Delta\}\), o ile \(\Gamma\cap\Delta=\emptyset\).
    \item \(\vdash\varphi\) zamiast \(\emptyset\vdash\varphi\).
  \end{itemize}

\item
  Na zbiorze sądów wprowadzamy relacje określające reguły wyprowadzania termów
    \begin{center}
    \begin{tabular}{ cc}
      {\begin{prooftree}
        \Hypo{ \Gamma, x^{\varphi} \vdash M^{\psi} }
        \Infer1[(Abs)]{\Gamma \vdash (\lambda\, x^{\varphi}.\, M)^{\varphi\to\psi}}
      \end{prooftree}},
      &
      {\begin{prooftree}
        \Hypo{\Gamma \vdash M^{\varphi \to \psi}} \Hypo{ \Gamma \vdash N^{\varphi}}
        \Infer2[(App)]{\Gamma \vdash (MN)^{\psi}}
      \end{prooftree}}.
      \end{tabular}
    \end{center}
    oraz wybieramy spośród sądów jeden aksjomat postaci \(\Gamma, x^\tau\vdash x^\tau\ (\mathrm{Var})\).

    Dowód sądu określamy analogicznie jak w logince NJ(\(\to\)).

Mówimy, że \(M\) jest \emph{termem} typu \(\tau\) w otoczeniu \(\Gamma\), jeśli istnieje dowód sądu \(\Gamma \vdash M^{\tau}\) w powyższym systemie dedukcyjnym.
\end{itemize}
  
\end{definicja}

