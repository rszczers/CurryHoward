Widzieliśmy, że konsekwentne \(\beta\)-redukowanie \(\lambda\)-termów nie zawsze prowadzi do uzyskania postaci normalnej. Fakt \ref{thm:cr_untyped} i~następujące po nim Wniosek \ref{thm:ch_wniosek1_untyped} i~Wniosek \ref{thm:ch_wniosek2_untyped} stwierdzają, że jeśli tylko mamy pewność, że \(\lambda\)-term ma postać normalną, to jest ona wyznaczona jednoznacznie i~doprowadzi nas do niej każda strategia normalizująca. Fakt \ref{thm:cr_untyped} to klasyczne twierdzenie, którego dowód można znaleźć w \cite[Rozdział 3.2]{Barendregt_1984} i ze względu na jego obszerność pozwalamy sobie je pominąć. 

\begin{fakt}(Twierdzenie Churcha-Rossera)\label{thm:cr_untyped}. 
%Jeśli \(M_1\to^{*}_\beta M_2\) i \(M_1\to^{*}_\beta M_3\), to istnieje \(M_4\in\mathbf{\Lambda}\) takie, że \(M_2\to^{*}_\beta M_4\) i \(M_3\to^{*}_\beta M_4\).
\(\beta\)-redukcja ma własność CR. 
\end{fakt}

\begin{wniosek}\label{thm:ch_wniosek1_untyped}
  Jeśli \(M=_\beta N\), to istnieje \(L\in\mathbf{\Lambda}\) takie, że \(M\to^{*}_\beta L\) oraz \(N\to^{*}_\beta L\).
\end{wniosek}
\begin{dowod}
  Niech \(M, N\in\mathbf{\Lambda}\) będą takie, że \(M=_\beta N\). Wówczas istnieje ciąg \(\lambda\)-termów \(M_0, M_1, \dots, M_{n-1}, M_n\) taki, że
  \[
    M_0\xleftrightarrow[\beta]{}M_1\xleftrightarrow[\beta]{}\ \dots\ \xleftrightarrow[\beta]{}M_{n-1}\xleftrightarrow[\beta]{}M_n,
  \]
  gdzie \(M_0\equiv M\) i \(M_n \equiv N\). Dowód przeprowadzimy przez indukcję względem \(n\). Rozważmy następujące przypadki:
  \begin{enumerate}[label={(\arabic*)}, ref={(\arabic*)}]
    \setlength\itemsep{0em}
    \item Jeśli \(n=0\), to \(M\equiv N\). Ustalając \(L\equiv M (\equiv N)\) w oczywisty sposób \(M\to^{*}_\beta L\) i \(N\to^{*}_\beta L\).
    \item Jeśli \(n=k>0\), to istnieje \(M_{k-1}\in\mathbf{\Lambda}\) takie, że 
   \[
    M\equiv M_0\xleftrightarrow[\beta]{}M_1\xleftrightarrow[\beta]{}\ \dots\ \xleftrightarrow[\beta]{}M_{k-1}\xleftrightarrow[\beta]{}M_k\equiv N
  \]
      Z założenia indukcyjnego wiemy, że istnieje \(L'\in\mathbf{\Lambda}\) takie, że \(M_0 \to^{*}_\beta L'\) i \(M_{k-1} \to^{*}_\beta L'\). Ponieważ \(\xleftrightarrow[\beta]{}\) jest symetryczna, rozważmy osobno przypadki \(M_{k-1}\to_\beta M_k\) i \(M_{k}\to_\beta M_{k-1}\).
  \begin{enumerate}[label={(\alph*)}, ref={(\alph*)}]
    \setlength\itemsep{0em}
    \item Jeśli \(M_{k-1}\to_\beta M_k\), to tym bardziej \(M_{k-1}\to^{*}_\beta M_k\). Ponieważ \(M_{k-1}\to^{*}_\beta L'\), to korzystając Faktu \ref{thm:cr_untyped} wnosimy, że istnieje \(L\in\mathbf{\Lambda}\) taki, że \(L'\to^{*}_\beta L\) i \(M_k \to^{*}_\beta L\), czyli
               \[ \begin{tikzcd}
                 M_0 \arrow{r} \arrow{dr}{*} & \dots \arrow{l} \arrow{r} &  M_{k-1} \arrow{l}{} \arrow{r}{} \arrow[swap]{dl}{*} & M_k \arrow{d}{*} \\%
                 & L' \arrow{rr}{*} & & L 
               \end{tikzcd}
               \] 
             \item Jeśli \(M_{k}\to_\beta M_{k-1}\), to ponieważ \(M_{k-1}\to^{*}_\beta L'\), natychmiast otrzymujemy, że \(M_k \to^{*}_\beta L'\). Ustalając \(L\equiv L'\) otrzymujemy tezę.
  \end{enumerate}
  \end{enumerate}
  \qed
\end{dowod}

\begin{wniosek}\label{thm:ch_wniosek2_untyped}
  % \begin{enumerate}[label={(\arabic*)}, ref={(\arabic*)}]
  % \setlength\itemsep{0em}
%  \item Jeśli \(N\) to postać normalna \(M\), to \(M\to^{*}_\beta N\).
  \item Każdy \(\lambda\)-term ma co najwyżej jedną postać normalną.
 % \end{enumerate}
\end{wniosek}
\begin{dowod}
  % \begin{enumerate}[label={(\arabic*)}, ref={(\arabic*)}]
  % \setlength\itemsep{0em}
  % \item Przypuśćmy, że \(N\in \mathrm{NF}_\beta\) i \(M=_\beta N\). Wówczas z Wniosku \ref{thm:ch_wniosek1_untyped} istnieje \(L\) takie, że \(M\to^{*}_\beta L\) i \(N\to^{*}_\beta L\). Ponieważ \(N\in \mathrm{NF}_\beta\) i \(N\to^{*}_\beta L\), to \(N\equiv L\). Ponieważ \(M\to^{*}_\beta L\), to \(M\to^{*}_\beta N\).
\item Przypuśćmy, że \(M\) ma dwie różne postacie normalne, \(N_1,\, N_2\). Wówczas na podstawie Definicji \ref{def:normalform}, \(M\to^{*}_\beta N_1\) i \(M\to^{*}_\beta N_2\). Z Faktu \ref{thm:cr_untyped} istnieje \(L\in\mathbf{\Lambda}\) taki, że \(N_1 \to^{*}_\beta L\) i \(N_2 \to^{*}_\beta L\). Ponieważ \(N_1,\,N_2\in \mathrm{NF}_\beta\), to \(N_1\equiv L \equiv N_2\).
  % \end{enumerate}
  \qed
\end{dowod}
