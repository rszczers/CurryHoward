\subsubsection{Uniwersalny polimorfizm}
Istnieje wiele wariantów wprowadzania typów prostych. Przedstawiony w tym rozdziale rachunek powszechnie nazywany jest \emph{stylem Currego}. Charakteryzuje go interesująca własność: poprawne termy mają jednoznacznie wyznaczony typ z dokładnością do podstawienia. Oznacza to, że zmienne wystepujące w typie każdego poprawnego termu są w istocie kwantyfikowane po zbiorze wszystkich typów prostych. Każdy taki typ nazywamy \emph{typem (uniwersalnie parametrycznie) polimorficznym}.

\begin{definicja}(Podstawienie typu)
Podstawienie typu \(\tau\) za zmienną typową \(p\) w typie \(\sigma\) nazywamy następującą funkcję :
\begin{align*}
  p[p/ \tau] &=\tau,\\
  q[p/ \tau] &=q,\ \text{jeśli}\ q\not\equiv p,\\
  (\sigma_1 \to \sigma_2) [p/\tau] &= \sigma_1 [p/\tau] \to \sigma_2 [p/\tau].
\end{align*}
  Jeśli \(\Gamma\) jest kontekstem, to przez \(\Gamma[p/\tau]\) oznaczamy podstawienie \(\tau\) za zmienną \(p\) dla wszystkich typów występujących w \(\Gamma\).
\end{definicja}

\begin{twierdzenie}\label{thm:polymorphism}
  Jeśli \(\Gamma\vdash_\mathbb{T} M:\sigma\), to \(\Gamma [p/\tau]\vdash_\mathbb{T} M:\sigma[p/\tau]\) dla dowolnego \(\tau\in\mathbb{T}\) i zmiennej \(p\in U\).
\end{twierdzenie}
\begin{dowod}
  Dowód przebiega przez indukcję względem długości wyprowadzenia \(\Gamma\vdash M:\sigma\). Szczegóły pomijamy. \qed
\end{dowod}

Mając na uwadze Twierdzenie \ref{thm:polymorphism} możemy wnioskować o wielu własnościach funkcji reprezentowanych przez \(\lambda\)-termy tylko na podstawie typu. Na przykład typowi \(\sigma\to\sigma\) odpowiada dokładnie jeden (z dokładnością do \(\alpha\)-konwersji) term \(\lambda x.\,x\) i reprezentuje on funkcję identycznościową; typom \(\sigma\to\rho\to\sigma\) i \(\sigma\to\rho\to\rho\) odpowiednio przypisać możemy wyłącznie projekcje fst i snd (określone w \ref{subsec:pairs}), zaś typowi \((\rho\to\tau) \to (\sigma\to\rho)\to\sigma\to\tau\) term \(\lambda g f x. g (f x)\) reprezentujący złożenie funkcji (Przykład \ref{ex:typing} \ref{ex:typing_2}), do do której wiemy z kolei, że jest łączna.
Jest to wraz ogólnej zależności: dysponując dowolnym typem polimorficznym otrzymujemy \emph{twierdzenie za darmo} \cite{Wadler1989} na temat termów, które mają ten typ. 

