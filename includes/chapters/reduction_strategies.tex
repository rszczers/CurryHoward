\subsubsection{Leniwa ewaluacja}\label{subsec:lazy_eval}
Wyróżniamy nastepujące rodzaje strategii redukcji.

\begin{definicja}
Strategię nazywamy:
\begin{enumerate}
\item \emph{normalną} (ang. \emph{normal-order}), gdy zawsze redukujemy redeksy pojedyńczo, rozpoczynając od najbardziej zewnętrznego redeksu od lewej strony wyrażenia.
\item \emph{aplikatywną} (ang. \emph{applicative-order}), jeśli zawsze redukujemy redeksy pojedyńczo, rozpoczynając od najbardziej zagnieżdżonego redeksu występującego najbardziej po lewej stronie wyrażenia.
\end{enumerate}
\end{definicja}

Klasyfikacja ta nie jest oczywiście wyczerpująca. W przeciwieństwie do strategii aplikatywnej (patrz Przykład \ref{ex:beta_reduction}\ref{ex:undeterministic_reduction_untyped}), okazuje się, że strategia normalna jest zawsze strategią normalizującą \cite[Rozdział 1.5]{Urzyczyn2006}. Niestety, narzuca ona w presymistycznym przypadku wykładniczą złożoność obliczeniową. Zobserwujmy następującą redukcję:

\begin{figure}[h]
\begin{align*}
\psset{arrows=->}
\pstree[levelsep=125pt,arrowsize=4pt 3]
{
  \TR{\begin{psmatrix}[colsep=0.4cm,rowsep=0.4cm,arrowsize=2pt 2]
        &    &          & @ \\
        &    & \lambda  &   & (*\ 5\ 4) \\
        &    & @           \\
        &  @ & & \circ         \\
    (+) &    & \circ           \\
    \everypsbox{\scriptstyle}%
    \psset{nodesep=3pt,arrows=->}
    \ncline{1,4}{2,3}
    \ncline{2,3}{3,3}
    \ncline{3,3}{4,2}
    \ncline{4,2}{5,1}
    \ncline{4,2}{5,3}
    \ncline{3,3}{4,4}
    \ncline{1,4}{2,5}
    \nccurve[ncurvA=1.5,ncurvB=4,angleA=-120,angleB=200]{5,3}{2,3}
    \nccurve[ncurvA=1,ncurvB=2.5,angleA=-60,angleB=-30]{4,4}{2,3}
  \end{psmatrix}}
}
{
  \TR{\begin{psmatrix}[colsep=0.4cm,rowsep=0.4cm,arrowsize=2pt 2]
        &    & @           \\
        &  @ & &(*\ 5\ 4)         \\
    (+) &    & (*\ 5\ 4)           \\
    \everypsbox{\scriptstyle}%
    \psset{nodesep=2pt,arrows=->}
    \ncline{1,3}{2,2}
    \ncline{1,3}{2,4}
    \ncline{2,2}{3,1}
    \ncline{2,2}{3,3}
  \end{psmatrix}}\nbput{\text{Normalna}}
  \TR{\begin{psmatrix}[colsep=0.4cm,rowsep=0.4cm,arrowsize=2pt 2]
        &    & @           \\
        &  @ & &(*\ 5\ 4)         \\
    (+) &    & &           \\
    \everypsbox{\scriptstyle}%
    \psset{nodesep=2pt,arrows=->}
    \ncline{1,3}{2,2}
    \ncline{1,3}{2,4}
    \ncline{2,2}{3,1}
    \nccurve[ncurvA=1.2,ncurvB=1,angleA=-90,angleB=-90]{2,2}{2,4}
  \end{psmatrix}}\naput{\text{Leniwa}}
}
\end{align*}
\caption{\(\beta\)-redukcja normalna wyrażenia (\ref{ex:normal_reduction}) (po lewej) i ze wspólnymi wierzchołkami (po prawej).}\label{fig:reduction_strategy}
\end{figure}


\begin{align*}
  \left(\lambda x.(+\ x\ x)\right)(*\ 5\ 4)\ &\to_\beta (+\ (*\ 5\ 4)(*\ 5\ 4))\tag{\(\blacktriangledown\)}\label{ex:normal_reduction}\\
  &\to_\beta (+\ 20\ (*\ 5\ 4))\\
  &\to_\beta (+\ 20\ 20)\\
  &\to_\beta 40.
\end{align*}

Widzimy, że redeksy są niepotrzebnie zwielokratniane, podczas gdy przy podejściu aplikatywnym zostałyby wpierw zredukowane. Obydwie strategie mają więc poważne wady.

Udanym kompromisem jest przeprowadzanie redukcji strategią normalną aż do uzyskania słabej czołowej postaci normalnej i wówczas kontynuowanie redukcji strategią aplikatywną. \cite[Rozdział 11.3]{PeytonJones:1987:IFP:1096899}. Zauważmy również, że używając grafowej reprezentacji \(\lambda\)-termów możemy zamiast powielania wierzchołków redukować graf ustawiając wskaźniki do aplikowanych termów (patrz Rysunek \ref{fig:reduction_strategy}). 

Strategie redukcji, które opierają się na współdzieleniu rezultatu redukcji nazywamy \emph{leniwymi} (ang. \emph{lazy evaluation}, także \emph{call-by-need}). Ich istotą jest redukowanie argumentów dokładnie raz. Przesądzają one o praktycznym zastosowaniu języków funkcyjnych w programowaniu.
