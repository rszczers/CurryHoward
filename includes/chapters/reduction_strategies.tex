\begin{definicja}
Strategię nazywamy:
\begin{enumerate}
\item \emph{normalną} (ang. \emph{normal-order}), gdy redukujemy redeksy pojedyńczo, rozpoczynając najbardziej zewnętrznego redeksu od lewej strony wyrażenia.
\item \emph{aplikatywną} (ang. \emph{applicative-order}), jeśli redukujemy redeksy pojedyńczo, rozpoczynając od najbardziej zagnieżdżonego redeksu występującego najbardziej po lewej stronie wyrażenia.
\end{enumerate}
\end{definicja}

W przeciwieństwie do strategii aplikatywnej (patrz Przykład \ref{ex:beta_reduction}\ref{ex:undeterministic_reduction_untyped}), okazuje się, że strategia normalna jest zawsze strategią normalizującą \cite[Rozdział 1.5]{Urzyczyn2006}. Niestety, narzuca ona w presymistycznym przypadku wykładniczą złożoność obliczeniową. Zobserwujmy następującą redukcję:
\begin{align*}
\left(\lambda x.(+\ x\ x)\right)(*\ 5\ 4)\ &\to_\beta (+\ (*\ 5\ 4)(*\ 5\ 4))\tag{R1}\\
  &\to_\beta (+\ 20\ (*\ 5\ 4))\\
  &\to_\beta (+\ 20\ 20)\\
  &\to_\beta 40.
\end{align*}
Widzimy, że redeksy są niepotrzebnie zwielokratniane, podczas gdy przy podejściu aplikatywnym zostałyby wpierw zredukowane. Obydwie strategie mają więc poważne wady.

Udanym kompromisem jest przeprowadzanie redukcji strategią normalną aż do uzyskania słabej czołowej postaci normalnej i wówczas kontynuowanie redukcji strategią aplikatywną. \cite[Rozdział 11.3]{PeytonJones:1987:IFP:1096899}. Używając grafowej reprezentacji \(\lambda\)-termów możemy zamiast powielania wierzchołków ustawiać wskaźnik do aplikowanego termu. 

\[
\psmatrix[colsep=0.4cm,rowsep=0.4cm]
    &    &          & @ \\
    &    & \lambda  &   & (*\ 5\ 4) \\
    &    & @           \\
    &  @ & & \circ         \\
(+) &    & \circ           \\
\endpsmatrix
\everypsbox{\scriptstyle}%
\psset{nodesep=3pt,arrows=->}
\ncline{1,4}{2,3}
\ncline{2,3}{3,3}
\ncline{3,3}{4,2}
\ncline{4,2}{5,1}
\ncline{4,2}{5,3}
\ncline{3,3}{4,4}
\ncline{1,4}{2,5}
\nccurve[ncurvA=1.5,ncurvB=4,angleA=-120,angleB=200]{5,3}{2,3}
\nccurve[ncurvA=1,ncurvB=2.5,angleA=-60,angleB=-30]{4,4}{2,3}
\]
