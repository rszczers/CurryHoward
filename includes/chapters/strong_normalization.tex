\subsubsection{Silna normalizacja}\label{subsec:strong_normalization}
Pokażemy, że wszystkie typowalne \(\lambda\)-termy redukują się do postaci \(\beta\)-normalnej przez skończony ciąg \(\beta\)-redukcji. Oznacza to, że nie ma możliwości otrzymania nieskończonego ciągu \(\beta\)-redukcji, tak jak to miało miejsce w Przykładzie \ref{ex:beta_reduction} \ref{ex:Y} \ref{ex:Delta} i to bez względu na przyjętą strategię redukcji. 

Ponieważ wszystkie \(\lambda\)-termy samoreplikujące się przy \(\beta\)-redukcji nie są typowalne, nie jest możliwe  w rachunku \(\lambda\) z typami prostymi reprezentowanie rekurencyjnych typów ADT w myśl podrozdziału \ref{sec:adt_recurrency}. Wynika to z faktu, że dodanie typów prostych do rachunku \(\lambda\) bez typów znacznie zmniejsza ekspresywność systemu, uniemożliwiając wyrażenie operacji rekursji prostej. Okazuje się, że stosujac reprezentację Churcha dla liczb naturalnych i utożsamiajac \(\lambda\)-termy za pomocą \(\beta\)-konwersji, rachunek \(\lambda\) z typami prostymi równoważny jest zbiorowi \emph{wielomianów rozszerzonych} \cite{DBLP:journals/corr/abs-cs-0701022}. Liczebnikom Churcha odpowiada wówczas typ postaci \((\sigma\to\sigma)\to\sigma\to\sigma\) i możliwe jest określenie na nich dodawania i mnożenia.

Ponieważ wszystkie ciągi redukcji są w tym systemie skończone, to relacja \(\beta\)-konwersji jest rozstrzygalna, wystarczy bowiem sprowadzić jej argumenty do postaci normalnej. Podejście to rodzi jednak nietrywialne problemy natury złożonościowej \cite[Podrozdział 3.7]{Urzyczyn2006}.
 
Opracowany tutaj dowód pochodzi z \cite[Dodatek A3]{Hindley:2008:LCI:1388400}. Polega on na:
\begin{enumerate}
  \setlength\itemsep{0em}
  \item Konstrukcji interpretacji dla typów prostych: termów redukowalnych.
  \item Wykazaniu, że każdy term redukowalny jest silnie normalizowalny.
  \item Wykazaniu, że każdy typowalny term jest redukowalny.
\end{enumerate}
Rozumowanie przedstawione w tym dowodzie, tzw. \emph{computability method} oryginalnie przypisywane W. Taitowi \cite{Tait67}, z odpowiednimi zmianami stosuje się w dowodach własności silnej normalizacji dla innych systemów typów \cite[Podrozdział 11.5]{Urzyczyn2006}.

\begin{definicja}(Termy redukowalne)
  Niech \(\Gamma\vdash_\mathbb{T} M:\sigma\). Powiemy, że \(M\) jest \emph{redukowalny} (także \emph{silnie obliczalny}), jeśli spełnia poniższe warunki:
  \begin{enumerate}[label=(R\arabic*), ref=(R\arabic*)]
    \item Jesli \(\sigma\) jest zmienną typową, to \(M\) jest silnie normalizowalny. Określamy:
      \[
        \llbracket \sigma \rrbracket = \mathrm{SN}.
      \]
    \item Jesli \(\sigma\) jest typem funkcyjnym postaci \(\sigma\equiv \rho\to\tau\), to dla wszystkich termów redukowalnych \(N\) takich, że \(\Gamma'\vdash_\mathbb{T}MN:\tau\), \(MN\) jest redukowalny. Określamy:
      \[
        \llbracket \rho \to \tau\rrbracket = \{M\ |\ \forall N (N\in\llbracket \rho \rrbracket)\implies MN\in\llbracket \tau \rrbracket\}.
      \]

  \end{enumerate}
\end{definicja}

\begin{lemat}\label{thm:sn_lemat1}
  Niech \(\tau\in\mathbb{T}\) bedzie dowolnym typem prostym. Wówczas:
  \begin{enumerate}[label=(\arabic*)]
    \setlength\itemsep{0em}
    \item \(\llbracket \tau \rrbracket  \subseteq \mathrm{SN}\).\label{def:l1_a}
    \item Jeśli \(N_1,\,N_2,\,\dots,\,N_k\in\mathrm{SN}\), to \(xN_1 N_2 \dots N_k \in \llbracket \tau \rrbracket\).\label{def:l1_b}
  \end{enumerate}
\end{lemat}
\begin{dowod}
  Dowód przeprowadzimy przez indukcję strukturalną względem typu \(\tau\). Mamy do rozważenia następujące dwa przypadki:
  \begin{enumerate}[label=(\alph*)]
    \setlength\itemsep{0em}
    \item \(\tau\) jest zmienną typową.
     \begin{enumerate}[label=(\arabic*)]
      \setlength\itemsep{0em}
     
        \item Wynika bezpośrednio z definicji \(\llbracket \tau \rrbracket\in \mathrm{SN}\). 
         
        \item Niech \(N_1,\,N_2,\,\dots,\,N_k \in \mathrm{SN}\). Wówczas \(N_1,\,N_2,\,\dots,\,N_k\in\mathrm{SN}\). Z definicji \(\llbracket \tau \rrbracket\) mamy, że \(xN_1 N_2 \dots N_k \in\llbracket \tau \rrbracket\).
      \end{enumerate}

    \item Przypuśćmy, że \(\tau=\sigma\to\rho\) oraz twierdzenie zachodzi dla \(\sigma\) i \(\rho\).
    \begin{enumerate}[label=(\arabic*)]
    \setlength\itemsep{0em}

    \item Niech \(M\in\llbracket \sigma \to \rho\rrbracket\) i niech \(x\) bedzie dowolną \(\lambda\)-zmienną. Z części \ref{def:l1_b} założenia indukcyjnego mamy \(x\in\llbracket \sigma \rrbracket\), zatem z definicji \(\llbracket \sigma\to\rho\rrbracket\) mamy \(Mx\in\llbracket p\rrbracket\). Ponieważ z części \ref{def:l1_a} założenia indukcyjnego \(\llbracket\rho\rrbracket\in\mathrm{SN}\), to \(Mx\in\mathrm{SN}\) i w konsekwencji \(\llbracket\sigma\to\rho\rrbracket\subseteq \mathrm{SN}\).
    \item Niech \(P\in\llbracket \sigma \rrbracket\). Wówczas z części \ref{def:l1_a} założenia indukcyjnego \(P\in\mathrm{SN}\). Chcemy pokazać, że \(xN_1 N_2 \dots N_k \in \llbracket \rho \rrbracket\). Z części \ref{def:l1_b} założenia indukcyjnego \[xN_1 N_2 \dots N_k N_{k+1} \in \llbracket \rho \rrbracket.\] Ustalając \(N_{k+1}\equiv P\) otrzymujemy tezę.
     \end{enumerate}
  \end{enumerate}
  \qed
\end{dowod}

\begin{lemat}\label{thm:sn_lemat2}
  Załóżmy, że:
  \begin{enumerate}[label=(\alph*)]
    \setlength\itemsep{0em}
    \item  \(M[x/N_0]N_1\dots N_k\in\mathrm{SN}\),\label{def:l2_a}
    \item  \(N_0\in \mathrm{SN}\).\label{def:l2_b}
  \end{enumerate}
  Wówczas \((\lambda x.\,M)N_0 N_1 \dots N_k \in \mathrm{SN}\).
\end{lemat}
\begin{dowod}
  (Ad  absurdum) Przypuśćmy,  że  \(P_0\equiv(\lambda  x.\, M)N_0  N_1
  \dots N_k \not\in\mathrm{SN}\). Wówczas istnieje nieskończony ciąg redukcji
  \[
    P_0\to P_1 \to \dots
  \]
  Każdy podterm \(\lambda\)-termu silnie normalizowalnego jest silnie
  normalizowalny. Ponieważ \(P_0\equiv M[x/N_0]N_0  N_1 \dots N_k
  \in\mathrm{SN}\), to \(M[x/N_0],\,N_0,\,  N_1,\, \dots,\, N_k
  \in\mathrm{SN}\). Na podstawie Lematu \ref{thm:sn_lem2'} mamy ponadto, że \(M\in\mathrm{SN}\). Wobec tego dla pewnego \(n\in\mathbb{N}\) redukcji ulega redeks czołowy:
  \begin{align*}
    P_n\equiv (\lambda x.\,M')N'_0 N'_1 \dots N'_k\to_\beta M'[x/N'_0]N'_0 N'_1\dots N'_k\equiv P_{n+1},
  \end{align*}
    gdzie \(M\to^{*}_\beta M'\) oraz \(N_i \to^{*}_\beta N'_i\) dla \(i\leq k\). Ale skoro tak, to prawdą jest również, że \(M[x/N_0]N_1\dots N_k \to^{*}_\beta P_{n+1}\), zaś \(M[x/N_0]N_1\dots N_k \in \mathrm{SN}\). Zatem \(P_{n+1}\in\mathrm{SN}\), co prowadzi do sprzeczonści.
  \qed
\end{dowod}

\begin{lemat}\label{thm:sn_lemat3}
  Załóżmy, że:
  \begin{enumerate}[label=(\alph*)]
    \setlength\itemsep{0em}
    \item  \(M[x/N_0]N_1\dots N_k\in \llbracket \tau \rrbracket\),\label{def:l3_a}
    \item  \(N_0\in \mathrm{SN}\).\label{def:l3_b}
  \end{enumerate}
  Wówczas \((\lambda x.\,M)N_0 N_1 \dots N_k \in \llbracket \tau \rrbracket \).
\end{lemat}
\begin{dowod}
  Dowód przebiega przez indukcję strukturalną względem \(\tau\). Rozważmy następujące przypadki:
  \begin{enumerate}[label=(\alph*)]
    \setlength\itemsep{0em}
    \item Jeśli \(\tau\) jest zmienną typową, to \(\llbracket \tau \rrbracket=\mathrm{SN}\). Wobec tego problem sprowadza się do Lematu \ref{thm:sn_lemat2}.  
    \item Przypuśćmy, że \(\tau\equiv \sigma\to\rho\) i niech \(M[x/N_0]N_1\dots N_k \in \llbracket \sigma\to \rho \rrbracket\). Wybierzmy dowolny \(P\in\llbracket\sigma\rrbracket\). Wówczas \(M[x/N_0] N_1 \dots N_k N_{k+1}\in\llbracket \rho\rrbracket\). Z założenia indukcyjnego mamy jednak, że 
      \(
        (\lambda x.\,M)N_0 N_1\dots N_k N_{k+1}\in\llbracket \rho \rrbracket.
      \)
      Wystarczy więc przyjąć \(N_{k+1}\equiv P\) i z definicji \(\llbracket \sigma \to \rho \rrbracket\) mamy, że \((\lambda x.\,M)N_0 N_1 \dots N_k \in \llbracket \sigma \to \rho \rrbracket\).
  \end{enumerate}
  \qed
\end{dowod}

\begin{definicja}
  Powiemy, że kontekst \(\Gamma=\{x_1:\sigma_1,\,x_2:\sigma_2,\dots ,\,x_n:\sigma_n\}\) \emph{spełnia} stwierdzenie \(M:\sigma\) i będziemy pisali \(\Gamma\models M:\sigma\), jeśli dla dowolnych \(N_1\in\llbracket \sigma_1 \rrbracket\), \(N_2\in\llbracket \sigma_2 \rrbracket\), \(\dots\), \(N_n\in\llbracket \sigma_n \rrbracket\) mamy, że:
  \[
    M[x_1/N_1,\,x_2/N_2,\,\dots,\,x_n/N_n]\in\llbracket \tau \rrbracket.
  \]
\end{definicja}
\begin{lemat}\label{thm:o_poprawnosci}
  Jeśli \(\Gamma\vdash_{\mathbb{T}}M:\tau\), to \(\Gamma\models M:\tau\).
\end{lemat}
\begin{dowod}
  Dowód będzie przebiegał przez indukcję względem wyprowadzenia \(\Gamma\vdash M:\tau\). Niech \(\Gamma=(x_1:\tau_1,\,x_2:\tau_2,\,\dots,\,x_n:\tau_n)\) będzie kontekstem dla którego istnieje wyprowadzenie \(J:\:\Gamma\vdash M:\tau\). Wybierzmy \(N_1\in\llbracket\tau_1 \rrbracket,\,N_2\in\llbracket\tau_2\rrbracket,\,\dots,\,N_n\in\llbracket\tau_n\rrbracket\). Rozważmy następujące przypadki:
  \begin{enumerate}[label=(\alph*)]
    \setlength\itemsep{0em}
    \item \(J\) jest konsekwencją reguły \emph{var}. Wówczas \(J\) jest postaci \(\Gamma \vdash x_i:\tau\) dla pewnego \(i\in\mathbb{N}\), \(1\leq i\leq n\), gdzie \(x_i:\tau\in\Gamma\). Stąd \(M[\vec{x}/\vec{N}]=x_i[x_i/N_i]=N_i\in\llbracket \tau \rrbracket\). Z dowolności \(N_i\), \(\Gamma\vDash M:\tau\).
    \item \(J\) jest konsekwencją reguły \emph{app}. Wówczas \(J\) jest postaci \(\Gamma\vdash PQ:\tau\). Z założenia indukcyjnego istnieje \(\sigma\in\mathbb{T}\) takie, że \(\Gamma\models P:\sigma\to\tau\) i \(\Gamma\models Q:\sigma\). Wobec tego \(P[\vec{x}/\vec{N}]\in\llbracket\sigma\to\tau\rrbracket\) i \(Q\llbracket \vec{x}/\vec{N} \rrbracket \in \llbracket \sigma \rrbracket\). Z definicji jednoczesnego podstawienia (Definicja \ref{def:simult_substitution}) mamy:
      \begin{align*}
        PQ[\vec{x}/\vec{N}]=P[\vec{x}/\vec{N}]Q[\vec{x}/\vec{N}]
      \end{align*}
      Z definicji \(\llbracket \sigma \to \tau \rrbracket\) wówczas \(M\in\llbracket \tau\rrbracket\).
    \item \(J\) jest konsekwencją reguły \emph{abs}. Wówczas \(J\) jest postaci \(\Gamma\vdash \lambda y.\,P:\sigma \to\rho\), gdzie \(y\not\in\mathrm{dom}\Gamma\). Z założenia indukcyjnego mamy, że \(\Gamma,y:\sigma\models P:\rho\). Oznacza to, że dla dowolnych \(N_1\in\llbracket\tau_1\rrbracket\), \(N_2\in\llbracket\tau_2\rrbracket\), \(\dots\), \(N_n\in\llbracket\tau_n\rrbracket\) mamy
    \begin{align} 
      \forall N\in\llbracket \sigma \rrbracket\, \left(
        P[\vec{x},y/\vec{N},N]\in\llbracket \rho \rrbracket \tag{\textasteriskcentered} \label{eq:poprawnosc}
      \right)
    \end{align} 
      Ustalmy \(P'\equiv P[y/y'][\vec{x}/\vec{N}]\), gdzie \(y'\not\in \mathrm{dom}\Gamma\) i \(y'\not \in \mathrm{FV}(N_i)\) dla \(i\in\mathbb{N}\), \(1\leq i\leq n\). Wówczas z \eqref{eq:poprawnosc}:
      \begin{align*}
        \forall N\in\llbracket\sigma\rrbracket\, \left(
        P'[y'/N]\in\llbracket \rho \rrbracket
        \right )
      \end{align*}
      Ustalmy \(N_0\in\llbracket \sigma \rrbracket\). Wówczas z części \ref{def:l1_a} Lematu \ref{thm:sn_lemat1} \(N_0\in\mathrm{SN}\). Wobec tego z Lematu \ref{thm:sn_lemat3} wnioskujemy, że:
      \begin{align}
        (\lambda y'.\,P')N_0\in\llbracket \rho \rrbracket \tag{\textasteriskcentered\textasteriskcentered}\label{eq:sn_2}
      \end{align}
      Zauważmy teraz, że ponieważ \(\forall i\ y_i\not\in\mathrm{FV}(N_i) \)
      \begin{equation}
        \begin{aligned}
        (\lambda y'.\,P')&=(\lambda y'.\,P[y/y'][\vec{x}/\vec{N}])\\
                         &=(\lambda y'.\,P[y/y'])[\vec{x}/\vec{N}]
                         =(\lambda y.\,P)[\vec{x}/\vec{N}] 
        \end{aligned} \tag{\textasteriskcentered\textasteriskcentered\textasteriskcentered}\label{eq:sn_3}
      \end{equation}
      Z \eqref{eq:sn_2} i \eqref{eq:sn_3} otrzymujemy     
      \[
        \left(
          (\lambda y.\,P)[\vec{x}/\vec{N}]
        \right)
        N_0 \in \llbracket \rho \rrbracket .
      \]
      Ponieważ \(N_0\in\llbracket \sigma \rrbracket\), to z definicji \(\llbracket \sigma \to \rho \rrbracket\) mamy, że 
      \[
        (\lambda y.\,P)[\vec{x}/\vec{N}]\in\llbracket \sigma \to \rho \rrbracket.
      \]
      Z dowolności \(\vec{N}\) otrzymujemy ostatecznie, że \(\Gamma\models \lambda y.\,P\).
  \end{enumerate}
  \qed
\end{dowod}
\begin{twierdzenie}(O silnej normalizacji)
  Jeżeli \(\Gamma\vdash_\mathbb{T}M:\tau\), to \(M\in\mathrm{SN_\beta}\).
\end{twierdzenie}
\begin{dowod}
  Na podstawie Lematu \ref{thm:o_poprawnosci}, jeśli \(\Gamma\vdash_\mathbb{T} M:\tau\), to \(M\in\llbracket \tau \llbracket\). Stosując Lemat \ref{thm:sn_lemat1} otrzymujemy tezę. \qed
\end{dowod}

Natychmiast widzimy, że własność silnej normalizacji pociąga za sobą własność słabej normalizacji, dlatego pomijamy dowód tej drugiej.
