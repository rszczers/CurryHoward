\subsection{Uogólniony system typów (GTS)}
  \subsubsection{Typowanie}
  Typizacja: \(\Gamma \vdash M : A\)\\

  Schematy reguł dzielimy na dwie grupy:

%  \begin{enumerate}
%      \item Reguły ogólne
%      \item Reguły wyróżniające
%  \end{enumerate}

\begin{enumerate}
\item  Reguły ogólne
  \begin{center}
  \begin{tabular}{r c c }

    \vspace{0.2cm}
    (Ax) &
    {\begin{prooftree}
      \Hypo{}
      \Infer1[]{<>\vdash *:\Box}
    \end{prooftree}} & \\
    \vspace{0.2cm}

    (Start) &
    {\begin{prooftree}
      \Hypo{\Gamma \vdash A:s}
      \Infer1[]{\Gamma, x:A \vdash x:A}
    \end{prooftree}} &
    \(x\not\in\Gamma\) \\
    \vspace{0.2cm}

    (Weak) &
    {\begin{prooftree}
      \Hypo{ \Gamma, A:B \vdash C:s }
      \Infer1[]{\Gamma, x:C \vdash A:B}
    \end{prooftree}} &
    \(x\not\in\Gamma\)\\
    \vspace{0.2cm}

%    (Type/Kind) &
%    {\begin{prooftree}
%      \Hypo{ \Gamma \vdash A:*} \Hypo{\Gamma, x:A \vdash B:s}
%      \Infer2[]{\Gamma \vdash (\Pi x:A.\, B):s }
%    \end{prooftree}} & \\
%    \vspace{0.2cm}

    (App) &
    {\begin{prooftree}
      \Hypo{\Gamma \vdash F:(\Pi x:A.\, B)} \Hypo{\Gamma \vdash a : A}
      \Infer2[]{\Gamma \vdash Fa:B[x:=a]}
    \end{prooftree}} & \\
    \vspace{0.2cm}

    (Abs) &
    {\begin{prooftree}
      \Hypo{\Gamma, x:A \vdash b:B } \Hypo{\Gamma \vdash (\Pi x:A.\, B) : s}
      \Infer2[]{\Gamma \vdash (\lambda x:A.\,b):(\Pi x:A.\, B)}
    \end{prooftree}} & \\
    \vspace{0.2cm}

    (Conv) &
    {\begin{prooftree}
      \Hypo{\Gamma \vdash A:B} \Hypo{\Gamma \vdash B':s } \Hypo{B =_{\beta} B'}
      \Infer3[]{\Gamma \vdash A:B'}
    \end{prooftree}} & \\

  \end{tabular}
  \end{center}

\item Reguły wyróżniające
  \begin{center}
  \begin{tabular}{r c c }
    (\((s_1, s_2)\)-reguła ) &
    {\begin{prooftree}
      \Hypo{ \Gamma \vdash A:s_1} \Hypo{\Gamma, x:A \vdash B:s_2}
      \Infer2[]{\Gamma \vdash (\Pi x:A.\, B):s_2 }
    \end{prooftree}} & \\

  \end{tabular}
  \end{center}
  \subsubsection{Redukcja}
\end{enumerate}
  \begin{center}
  \begin{tabular}{r | l | c c c c}
    0 & \(\lambda_{\to}\)                 & \((*,\,*)\) \\
    1 & \(\lambda 2\)                     & \((*,\,*)\) & \((\Box,\,*)\) \\
    2 & \(\lambda P\)                     & \((*,\,*)\) & & \((*,\,\Box)\) \\
    1+2 & \(\lambda P2\)                  & \((*,\,*)\) & \((\Box,\,*)\) & \((*,\,\Box)\) \\
    3 & \(\lambda \underline{\omega}\)    & \((*,\,*)\) & & & \((\Box,\,\Box)\)\\
    1+3 & \(\lambda \omega\)              & \((*,\,*)\) & \((\Box,\,*)\) & & \((\Box,\,\Box)\)\\
    2+3 & \(\lambda P\underline{\omega}\) & \((*,\,*)\) & & \((*,\,\Box)\) & \((\Box,\,\Box)\) \\
    1+2+3 & \(\lambda C\)                 & \((*,\,*)\) & \((\Box,\,*)\) & \((*,\,\Box)\) & \((\Box,\,\Box)\) \\
  \end{tabular}

  \begin{itemize}
    \item \((*,*)\) - termy zależne od termów (typy funkcyjne),
    \item \((\Box,*)\) - termy zalezne od typów (typy polimorficzne),
    \item \((*,\Box)\) - typy zależne od termów (typy zależne),
    \item \((\Box, \Box)\) - typy zależne od typów (rodziny typów, higher-kinded types).
  \end{itemize}
  \end{center}


