\section{\(\mathrm{IPC}(\rightarrow)\)}
\subsection{Język}

\begin{definicja}$ $\newline
  \begin{itemize}
    \item  Zbiorem \(\Phi_{\to}\) formuł \(\mathrm{IPC}(\rightarrow)\) nazywamy język generowany przez gramatykę
    \begin{align*}
    %\Phi &:= \mathrm{V}\ |\ \left ( \Phi \land \Phi \right )\ |\ \left ( \Phi \lor \Phi \right ) \ |\ \left ( \Phi \rightarrow \Phi \right )\ |\ \bot\\
      \Phi_{\to} &:= \mathrm{V}\ |\ \left ( \Phi_{\to} \rightarrow \Phi_{\to} \right )\ |\ \bot\\
    \mathrm{V} &:= \mathrm{p}\ |\ \mathrm{V'}
    \end{align*}

    \item Wyrażenia powstałe z produkcji V nazywamy \emph{zmiennymi zdaniowymi}. Zmienne zdaniowe oraz \(\bot\) są formułami \emph{atomowymi}. Pozostałe wyrażenia nazywamy formułami \emph{złożonymi}.


\item  Konwencje: \begin{enumerate}
    \item W języku podmiotowym wprowadzamy następujące oznaczenia
    \begin{align*}
      \lnot \varphi &:=\ \left\ulcorner\varphi\rightarrow\bot\right\urcorner\\
    %  \left\ulcorner\varphi \leftrightarrow \psi \right\urcorner\ &:=\ \left\ulcorner\left(\varphi\rightarrow\psi\right)\land\left(\psi\rightarrow\varphi\right)\right\urcorner\\
      \top &:=\ \left\ulcorner\bot\rightarrow\bot\right\urcorner
    \end{align*}

    \item Zamiast \(\mathrm{p'},\, \mathrm{p''},\, \mathrm{p'''},\, \dots\) używamy kolejno liter \(\mathrm{p},\, \mathrm{q},\, \mathrm{r},\, \dots\)
    \item Zmienne metasyntaktyczne oznaczamy późniejszymi literami greckiego alfabetu, tj. \(\varphi,\, \psi,\, \dots\)
    \item \(\rightarrow\) jest łączna w prawo.
    %\item \(\lnot\) ma najwyższy priorytet, \(\rightarrow\) – najniższy, \(\land\), \(\lor\) – równy sobie.
    \item \(\lnot\) ma najwyższy priorytet, \(\rightarrow\) – najniższy.
    \item Pomijamy najbardziej zewnętrzne nawiasy.
  \end{enumerate}

\item
  Każdą parę \((\Gamma, \varphi)\in\mathcal{P}\left(\Phi_{\to}\right)\times\Phi_{\to}\), gdzie \(\Gamma\) jest zbiorem skończonym nazywamy \emph{sądem} (\emph{asercją}) i oznaczamy \(\Gamma\vdash\varphi\).

  Piszemy: \begin{itemize}
    \item \(\varphi_{1},\ \varphi_{2}\vdash\psi\) zamiast \(\{\varphi_{1},\ \varphi_{2}\}\vdash\psi\),
    \item \(\Gamma, \varphi\) zamiast \(\{\Gamma\cup \varphi\}\),
    \item \(\Gamma, \Delta\) zamiast \(\{\Gamma\cup \Delta\}\),
    \item \(\vdash\varphi\) zamiast \(\emptyset\vdash\varphi\).
  \end{itemize}

  \item Na zbiorze sądów \(\mathcal{P}(\Phi_{\to})\times\Phi_{\to}\) wprowadzamy relacje okreslające reguły wyprowadzania
  \begin{center}
    \begin{tabular}{ ccc}
    {\begin{prooftree}
      \Hypo{ \Gamma, \varphi \vdash \psi }
      \Infer1[(\(\rightarrow\)I)]{\Gamma \vdash \varphi \to \psi}
    \end{prooftree}},
    &
    {\begin{prooftree}
      \Hypo{\Gamma \vdash \varphi \to \psi} \Hypo{ \Gamma \vdash \varphi}
      \Infer2[(\(\rightarrow\)E)]{\Gamma \vdash \psi}
    \end{prooftree}}.
    \end{tabular}
  \end{center}
  oraz wybieramy spośród nich jeden aksjomat postaci \(\Gamma, \varphi\vdash\varphi\ (\mathrm{Ax})\).

  \item \emph{Dowodem} sądu \(\Gamma \vdash \varphi\) nazywamy skończone drzewo sądów spełniające poniższe warunki:
    \begin{enumerate}
      \item W korzeniu drzewa znajduje się dowodzony sąd \(\Gamma \vdash \varphi\).
      \item Liście są \emph{aksjomatami}, tj. sądami postaci \(\Gamma, \varphi \vdash \varphi\).
      \item Każdego rodzica można otrzymać z jego dzieci przez zastosowanie którejś z reguł wyprowadzania nowych sądów.
  \end{enumerate}

      Jeśli istnieje dowód sądu \(\Gamma\vdash\varphi\) to mówimy, że formuła \(\varphi\) jest \emph{wyprowadzalna} ze zbioru \emph{przesłanek} \(\Gamma\) i piszemy \(\Gamma\vdash_{N}\varphi\). Formułę \(\varphi\) nazywamy wówczas \emph{tezą} systemu NJ(\(\to\)).

  \end{itemize}
\end{definicja}

\subsection{Semantyka}
\begin{twierdzenie}{(O pełności)}
  System dedukcyjny \(\mathrm{NJ}(\to)\) jest pełny względem modeli Kripkego.
\end{twierdzenie}
