\epigraph{
For a large class of cases of the~employment of the~word ‘meaning’ -- though not for all -- this word can be explained in this way: the~meaning of a word is its use in the language. \cite{Wittgenstein53}
}{\emph{\small{L. Wittgenstein}}}

\epigraph{The meaning of a proposition is~determined by (...) what counts as a
verification of it. \cite{Lof_96}}{\emph{\small{{P. Martin-Löf}}}}


\section*{Wstęp}
\subimport{includes/chapters/}{denotational_semantics.tex}

Obydwa ujęcia są wspólne dla programowania funkcyjnego. Refleksja nad rachunkiem \(\lambda\) D. Scotta i Ch. Strachey'a prowadziła do powstania semantyki denotacyjnej dla rachunku \(\lambda\) i w ogóle pozwoliła wyodrębnić się dziedzinie formalnej semantyki języków programowania. Przyjmując drugą perspektywę, wynosimy z bogatych teorii matematycznych abstrakcje, które kształtują techniki, którymi przetwarzamy informacje: funkcje wyższego rzędu, leniwa ewaluacja, dopasowywanie wzorców, różnorodne metody abstrakcji danych. 
Celem tej pracy jest wyłożenie minimalnych podstaw teoretycznych koniecznych do zrozumienia powyższych haseł. Bardzo konkretnym zamiarem tej pracy jest przygotowanie bazy wiedzy koniecznej do rozpoczęcia studiowania kompilatora GHC języka Haskell.

Pierwszy rozdział o rachunku \(\lambda\) bez typów to podstawa, na której opierały będą się pozostałe. Omówiona jest składnia rachunku \(\lambda\) i jego operacyjny charakter oraz elementy semantyki denotacyjnej. Rozdział kończy omówienie alternatywnej do kodowania Churcha metody reprezentowania struktur danych -- kodowania Scotta.

Drugi rozdział uzupełnia rachunek \(\lambda\) o typy. Omawiamy własności tego systemu: związki jego operacyjnego charakteru z typowaniem i ewaluacją. Główny wynik to własność silnej normalizacji.

W pracy przechodzimy z systemu nietypowanego, przez system w którym termy mają arbitralny (dynamiczny) charakter, do systemu, w którym typy wyrażeń są z góry ustalone (statyczne). System F w Rozdziale \ref{sec:system_f} zarysowany jest tak, by wskazać różnice pomiędzy systemami w stylu Currego i systemami w stylu Churcha. Składnia systemu wybrana została tak, aby zasugerować możliwe rozszerzenia systemu i z łatwością odnieść go do tzw. kostki \(\lambda\) H. P. Barendregta. Informacje tu zgromadzone są punktem wyjścia do zrozumienia technikaliów związanymi z kompilowaniem języka Haskell do języka pośredniego, który oparty jest w przeważającej mierze na Systemie F w stylu Churcha. 
