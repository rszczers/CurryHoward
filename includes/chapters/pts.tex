\section{Uogólnione systemy typów}
%Od tej pory rozważać będziemy wyłącznie systemy w stylu Churcha.

System typów, który był przedmiotem Rozdziału \ref{sec:simple_types} jest najbardziej elementarnym przypadkiem typowanego rachunku \(\lambda\)\footnote{Patologicznym przypadkiem jest rachunek \(\lambda\) bez typów, jeśli przyjmiemy, że wszystkie wyrażenia mają w nim dokładnie jeden typ. Argument ten często podejmowany jest na rzecz statycznie typowanych języków programowania.}. W literaturze często spotyka się być może jeszcze prostszy, równoważny wariant typów prostych, w którym wszystkie typy buduje się wyłącznie z jednej stałej typowej. Pod pojęciem typów prostych rozumie się także szereg rozszerzeń przedstawionego przez nas systemu. Ich celem jest zwiększenie ekspresji, np. o odpowiednie konstrukcje dla par, w których pierwszy element może być innego typu niż drugi. Konstrukcja ta jest bowiem niemożliwa w naszym, elementarnym ujęciu.

Rozszerzenia takie mają na ogół szczególny cel praktyczny: rozszerzenie o typ dla par umożliwiają na przykład elegancką prezentację analogii między intuicjonistycznym rachunkiem zdań, typowanym rachunkiem \(\lambda\) i kategoriami kartezjańsko domkniętymi, znanej szerzej jako izomorfizm Currego-Howarda-Lambeka \cite[Rozdział 3.1]{Girard:1989:PT:64805}. Nie wpływa to jednak na samą istotę typowania.  

Na kanwie zaproponowanej przez H. P. Barendregta w \cite[Rozdział 5]{Barendregt_1992} klasyfikacji rozszerzeń rachunku \(\lambda\) z typami prostymi (tzw. \emph{kostki \(\lambda\)}, Rysunek \ref{fig:lambda-cube}), rozdział ten poświęcimy omówieniu wzajemnych zależności jakie mogą łączyć \(\lambda\)-termy i typy. Zajmować będziemy się wyłącznie systemami w stylu Churcha.

\begin{figure}[!h]
  \centering
  \begin{tikzpicture}
  \matrix (m) [matrix of math nodes,
  row sep=3em, column sep=3em,
  text height=1.5ex,
  text depth=0.25ex]{
              & \lambda\omega             &              & \lambda\Pi\omega             \\
  \lambda 2   &                           & \lambda\Pi 2                                \\
              & \lambda\underline{\omega} &              & \lambda\Pi\underline{\omega} \\
  \lambda{\to}&                           & \lambda\Pi  \\
  };
  \path[-{Latex[length=2.5mm, width=1.5mm]}]
  (m-1-2) edge (m-1-4)
  (m-2-1) edge (m-2-3)
          edge (m-1-2)
  (m-3-2) edge (m-1-2)
          edge (m-3-4)
  (m-4-1) edge (m-2-1)
          edge (m-3-2)
          edge (m-4-3)
  (m-3-4) edge (m-1-4)
  (m-2-3) edge (m-1-4)
  (m-4-3) edge (m-3-4)
          edge (m-2-3);
  \end{tikzpicture}
  \caption{Poszczególne systemy klasyfikacji H. Barendregta; kierunek krawędzi \(\to\)  oznacza relację \(\subseteq\).}\label{fig:lambda-cube}
\end{figure}

\subsection{System \(\lambda{\to}\)}
Przez system \(\lambda{\to}\) rozumiemy rachunek \(\lambda\)  z typami prostymi w stylu Churcha (Rozdział \ref{subsec:church_style}). W rachunku tym mamy do czynienia z wyraźnym podziałem na obiekty dwóch rodzajów: \(\lambda\)-termy i typy. \(\lambda\)-termy możemy przekształcać dwoma dualnymi operacjami: \(\lambda\)-abstrakcją i aplikacją. Rezultat operacji zależy od wyboru zmiennej wolnej, którą chcemy wyabstrahować z termu albo wyboru termu, który chcemy zaaplkować do innego termu, odpowiednio. Innych możliwości nie ma. Dlatego mówimy, że w systemie \(\lambda{\to}\) termy \emph{zależą} od termów. Ponieważ abstrahowanie przebiega wyłącznie po zbiorze \(\lambda\)-zmiennych, mówimy, że zależność jest \emph{pierwszego rzędu}.

\begin{przyklad}
  \begin{enumerate}[label=(\alph*)]
    \item
  Zauważmy, że nie istnieje jeden typ dla reprezentacji funkcji identycznościowej. Jeśli \(nat\) jest stałą typową, którą reprezentujemy liczby naturalne, to identyczność na zbiorze liczb naturalnych będziemy reprezentowali termem \(\lambda x:nat.\,x\), na zbiorze funkcji \(\mathbb{N}\to\mathbb{N}\), \(\lambda x:\mathrm{nat}\to\mathrm{nat}.\,x\) i tak dalej.
  Aby okreslić ogólną postać identyczności, musimy móc abstrahować po zbiorze typów, czyli parametryzować postać termu typem. Własność tę miał w pewnym sensie rachunek \(\lambda\) w stylu Currego (Podrozdział \ref{subsec:polymorphism}). 
    \item
  \end{enumerate}
\end{przyklad}

