\subsection{Modele}
Ustalmy wpierw, że nie możemy naiwnie interpretować \(\lambda\)-termów jako funkcji i aplikacji argumentów do funkcji. Przypuśćmy bowiem, że w pewnej interpretacji \(\llbracket M \rrbracket=f_M\), gdzie \(f_M \in A\to B\) dla pewnych zbiorów \(A\) i \(B\). Wówczas \(\llbracket M M\rrbracket=\nobreak f_M(f_M)\), a zatem \(f_M\in A\). Oznacza to, że funkcja \(f_M\) jest elementem swojej własnej dziedziny, czyli istnieje nieskończony zstępujący ciąg zbiorów \[A\times f(A)\ \supset\ A\times f(A)\times f(A)\ \supset\ A\times f(A)\times f(A)\times f(A)\ \supset\ \dots\] Istnienie takiego ciągu narusza aksjomat ufundowania na gruncie aksjomatyki ZFC, a to wyklucza możliwość określenia takich modeli.

\subsubsection{Model kombinatoryczny}
\begin{definicja}(Struktura aplikatywna)
Parę \((D,\, \bullet)\), gdzie \(D\) jest zbiorem zawierającym przynajmniej dwa elementy i symbol \(\bullet\) oznacza działanie binarne na \(D\), nazywamy \emph{strukturą aplikatywną}.
\end{definicja}

\begin{definicja}
Niech \((D,\,\bullet)\) będzie strukturą aplikatywną i niech \(a,\,b\in D\). Określamy relację: 
\begin{align*}
a \sim b \quad \Leftrightarrow\quad \forall d \in D \left(a \bullet b = b \bullet d\right).
\end{align*}
Powiemy, że \(a\) jest \emph{ekstensjonalnie równoważne} \(b\), jeśli \(a\sim b\).
\end{definicja}


\begin{definicja}
\begin{enumerate}
\item
Powiemy, że term \(M\) jest \emph{kombinacją zmiennych}, jeśli \(M\equiv x\) dla pewnej zmiennej \(x\) albo \(M\equiv (PQ)\) dla pewnych kombinacji zmiennych \(P\) i \(Q\). 
\item 
Niech \(M\) będzie kombinacją zmiennych taką, że \(\mathrm{FV}(M)=\{x_1,\,x_2,\,\dots,\,x_n\}\). Względem \(M\) określamy funkcję \(f_M: D^n\to D\) następującym wzorem:

% \begin{align*}
% f_M (d_1,\,d_2,\,\dots,\,d_n) = \begin{cases}
% d_k,\ \text{jeśli}\ M\equiv x_k \text{dla pewnego}\ 1\leq k \leq n,\\
% f_Q (d_1,\,d_2,\,\dots,\,d_n) \bullet f_Q (d_1,\,d_2,\,\dots,\,d_n)\  \text{jeśli}\ M\equiv (QR)\ \text{dla pewnych kombinacji zmiennych QR}.
% \end{cases}
% \end{align*}
\begin{align*}
f_M (d_1,\,d_2,\,\dots,\,d_n) = \begin{cases}
d_k,\ &\text{jeśli}\ M\equiv x_k, \\
f_Q (d_1,\,d_2,\,\dots,\,d_n) \bullet f_Q (d_1,\,d_2,\,\dots,\,d_n),\ & \text{jeśli}\ M\equiv (PQ).
\end{cases}
\end{align*}

\item
Powiemy, że struktura aplikatywna \((D,\,\bullet)\) jest \emph{kombinatorycznie zupelna}, jeśli dla każdej kombinacji zmiennych \(M\) istnieje \(a_M\in D\) taki, że dla wszystkich \(d_1\),\(d_2\),\(\dots\),\(d_n\in D\)
\[
  a_M\bullet d_1 \bullet d_2 \bullet \dots \bullet d_n = f_M(d_1,\, d_2,\,\dots,\,d_n).
\]
\end{enumerate}
\end{definicja}

\begin{twierdzenie}
Struktura aplikatywna \((D,\,\bullet)\) jest kombinatorycznie zupełna wtedy i tylko wtedy, gdy istnieją takie \(k,\,s\in D\), że dla wszystkich \(a,\,b,\,c\in D\) mamy: 
\begin{align*}
k\bullet a\bullet b = a\quad \text{oraz}\quad s\bullet a \bullet b \bullet c = a\bullet c \bullet (b \bullet c).
\end{align*}
\end{twierdzenie}

\subsubsection{Model Scotta \(D_\infty\)}
\paragraph{Elementy teorii porządku}
Niech \((D,\sqsubseteq)\) będzie zbiorem częściowo uporzadkowanym. Powiemy, że \(b\in D\) jest elementem \emph{najmniejszym}, jesli \(b\sqsubseteq d\) dla wszystkich \(d\in D\). Element ten, o ile istnieje, wyznaczony jest jednoznacznie i będziemy oznaczać go symbolem \(\perp\). Niech \(X\subset D\). \emph{Kresem górnym} \(X\) nazywamy element \(u\in D\) taki, że \(x\sqsubseteq u\) dla wszystkich \(x\in X\). \emph{Supremum} zbioru \(X\) nazywamy element \(\ell\in D\) taki, że \(\ell\) jest ograniczeniem górnym \(X\) i \(\ell\sqsubseteq u\) dla wszystkich ograniczeń górnych \(u\) zbioru \(X\). Element taki, o ile istnieje, będziemy oznaczali symbolem \(\bigsqcup X\). Podzbiór \(X\subset D\) nazywamy \emph{skierowanym}, jeśli \(X\neq\emptyset\) i dla dowolnych \(x, y\in X\) istnieje \(z\in X\) taki, że \(x\sqsubseteq z\) i \(y\sqsubseteq\). 

\begin{definicja}(Zupełny porządek częściowy)
Porządek częściowy \((D,\sqsubset)\) taki, że
\begin{enumerate}
  \item \(\perp\in D\) oraz
  \item każdy skierowany poddzbiór \(X\subset D\) ma supremum \((\bigsqcup \in X)\)
\end{enumerate}
nazywamy \emph{zupełnym porzadkiem częściowym} (w skrócie: cpo).
\end{definicja}

Ustalmy, że jeśli \(D'\), \(D''\), \(\dots\)  są cpo, to odpowiadające im porządki częściowe będziemy notowali symbolami \(\sqsubseteq'\), \(\sqsubseteq''\), \(\dots\)

\begin{definicja}(Monotonoczność, ciągłość)\label{def:m_cont}
Niech \(D\) i \(D'\) bedą cpo i \(\varphi: D\to D'\) będzie funkcją.
\begin{enumerate}
\item Powiemy, że \(\varphi\) jest \emph{monotoniczna}, jeśli \(\varphi(a) \sqsubseteq' \varphi(b)\) dla \(a\sqsubseteq b\).
\item Powiemy, że \(\varphi\) jest \emph{ciągła}, jeśli \(\varphi(\bigsqcup X) = \bigsqcup \varphi (X).\)\label{def:m_cont_2}
\end{enumerate}
Symbolem \([D\to D']\) oznaczamy zbiór wszystkich funkcji ciągłych ze zbioru \(D\)~do~\(D'\).
\end{definicja}

\begin{twierdzenie}
Jeśli \(D\) i \(D'\) są cpo, to \(([D\to D'], \preceq)\) jest cpo, gdzie
\[
\varphi \preceq \psi\quad \Leftrightarrow\quad \forall d\in D \left(\varphi(d) \sqsubseteq' \psi(d)\right).
\]
\end{twierdzenie}

\begin{definicja}(Projekcja)
Niech \(D\) i \(D'\) będą cpo. \emph{Projekcją} z \(D'\) do \(D\) nazywamy parę \((\varphi, \psi)\) funkcji \(\varphi \in [D\to D']\), \(\psi \in [D'\to D]\) takich, że
\begin{align}
\psi\circ \varphi = I_D\ \text{oraz}\ \varphi\circ \psi \preceq I_{D'} \tag{\textasteriskcentered},
\end{align}
gdzie przez \(I_D\) i \(I_{D'}\) oznaczamy funkcję identycznościową na zbiorze \(D\) i \(D\), odpowiednio.
\end{definicja}

\begin{definicja}
Dla \(m,\,n\geq 0\) określamy \(\varphi_{mn}:D_m \to D_n\) w następujący sposób:
\begin{align}
\varphi_{mn} =
\begin{cases}
\varphi_{n-1} \circ \varphi_{n-2} \circ \dots \varphi_{m+1} \circ \varphi_m, & \text{jeśli}\ m\leq n,\\
I_{D_n}, & \text{jeśli}\ m=n,\\
\psi_n \circ \psi_{n+1} \circ \dots \circ \psi_{m-2}\circ \psi_{m-1} & \text{jeśli} m\geq n.
\end{cases}
\end{align}
\end{definicja}

\begin{lemat}
Niech \(m,\,n\geq 0\). Wówczas
\begin{enumerate}
\item \(\varphi_{mn}\in [D_m\to D_n]\),
\item jeśli \(m\leq n\), to \(\varphi_{nm}\circ \varphi_{mn} = I_{D_m}\),
\item jeśli \(m>n\), to \(\varphi_{nm}\circ \varphi_{mn} \preceq I_{D_m}\),
\item jeśli \(m<n\), to \((\varphi_{mn},\varphi_{nm})\) jest projekcją z \(D_n\) do \(D_m\),
\item jeśli \(m<k<n\) lub \(n<k<m\), to \(\varphi_{kn}\circ\varphi{mk}=\varphi_{mn}\).
\end{enumerate}
\end{lemat}

\paragraph{Konstrukcja \(D_\infty\)}

\begin{definicja}
Niech \(D_\infty\) oznacza zbiór wszystkich nieskończonych ciągów postaci
\[
d=(d_0,\,d1,\,\dots)
\]
takich, że dla wszystkich \(n\geq 0\) mamy, że \(d_n\in D_n\) oraz \(\psi_n (d_{n+1}) = d_n\). Przez \(d_n\) oznaczać będziemy \(n\)-ty element ciągu \(d\). Jeśli \(X\subset D_\infty\), to okreslamy \(X_n=\{ d_n\ |\ d\in X\).

Na \(D_\infty\) określamy relację \(\sqsubseteq\) w nastepujący sposób:
\[
(d_0,\,d_1,\,\dots) \sqsubseteq (d_0',\,d_1',\,\dots) \quad \Leftrightarrow\quad  \forall n\geq 0\  (d_n\sqsubseteq d_n') 
\]
\end{definicja}

\begin{twierdzenie}
\begin{enumerate}
\item \(D_\infty\) jest cpo.
\item \(D_\infty\) zawiera element najmniejszy \(\perp=(\perp_0,\,\perp_1,\,\dots)\), gdzie przez \(\perp_n\) oznaczamy najmniejszy element \(D_n\).
\item Supremum każdego skierowany zbioru \(X\subset D_\infty\) ma postać
\[
\bigsqcup X = (\bigsqcup X_0,\,\bigsqcup X_1,\,\dots)
\]
\end{enumerate}
\end{twierdzenie}

\begin{definicja}
Dla \(n\geq 0\) okreslamy funkcje \(\varphi_{n\infty}:D_n\to D_\infty\) oraz \(\varphi_{\infty n}: D_\infty \to D_n\), gdzie 
\begin{align*}
\varphi_{n\infty}(d) &= (\varphi_{n0}(d),\,\varphi_{n1}(d),\,\dots)\ \text{dla}\ d\in D_n,\\
\varphi_{\infty n}(d) &= d_n.
\end{align*}
\end{definicja}

\begin{twierdzenie}
Niech \(m,\,n\geq 0\), \(m\leq n\) i \(a,\,b\in D_\infty\). Wówczas:
\begin{enumerate}
\item \((\varphi_{n\infty},\,\varphi{\infty n})\) jest projekcją z \(D_\infty\) do \(D_n\),
\item \(\varphi_{mn}(a_m)\sqsubseteq a_n\),
\item \(\varphi_{m\infty}(a_m)\sqsubseteq \varphi_{n\infty}(a_n)\),
\item \(a=\bigsqcup_{n\geq 0}\varphi_{n\infty}(a_n),\),
\item \(\varphi_{n\infty}(a_{n+1}(b_n))\sqsubseteq\varphi_{(n+1)\infty}(a_{n+2}(b_{n+1}))\).
\end{enumerate}
\end{twierdzenie}

\begin{definicja}
Określamy
\begin{align*}
a \bullet b = \bigsqcup\left\{\varphi_{n\infty}(a_{n+1}(b_n))\ |\ n\geq 0\right\}\tag{\textasteriskcentered}
\end{align*}
\end{definicja}

