\subsection{Model Scotta \(D_\infty\)}
W rozdziale tym przybliżymy konstrukcję modelu Scotta \(D_\infty\) zgodnie z \cite[Rozdział 16]{Hindley:2008:LCI:1388400}. Ze względu na obszerność i topologiczny charakter modelu szczegółowe wyprowadzenia zostaną pominięte.

Ustalmy wpierw, że nie możemy naiwnie interpretować \(\lambda\)-termów jako funkcji i aplikacji argumentów do funkcji. Przypuśćmy bowiem, że w pewnej interpretacji \(\llbracket M \rrbracket=f_M\), gdzie \(f_M \in A\to B\) dla pewnych zbiorów \(A\) i \(B\). Wówczas \(\llbracket M M\rrbracket=\nobreak f_M(f_M)\), a zatem \(f_M\in A\). Oznacza to, że funkcja \(f_M\) jest elementem swojej własnej dziedziny, czyli istnieje nieskończony zstępujący ciąg zbiorów \[A\times f(A)\ \supset\ A\times f(A)\times f(A)\ \supset\ A\times f(A)\times f(A)\times f(A)\ \supset\ \dots\] Istnienie takiego ciągu narusza aksjomat ufundowania na gruncie aksjomatyki ZFC, a to wyklucza możliwość określenia takich modeli.

% \subsubsection{Wprowadzenie}
\begin{definicja}[Struktura aplikatywna]
Parę \((D,\, \bullet)\), gdzie \(D\) jest zbiorem zawierającym przynajmniej dwa elementy i w którym symbol \(\bullet\) oznacza działanie binarne na \(D\), nazywamy \emph{strukturą aplikatywną}.
\end{definicja}

% \begin{definicja}(Model syntaktyczny)%3.3
%   Trójkę \((D,\,\bullet,\,\llbracket\,\rrbracket)\), gdzie \((D,\,\bullet)\) jest strukturą aplikatywną, \(\llbracket\,\rrbracket\) jest funkcją przyporządkowującą elementy 
%   \(\llbracket M\rrbracket \in D\) termowi \(M\) przy wartościowaniu \(\rho\), nazywamy \emph{\(\lambda\)-modelem}, jeśli spełnia poniższe własności:
%   \begin{enumerate}
%     \item \(\forall x \in V\ \left(\llbracket x \rrbracket_\rho = \rho(x)\right)\).
%     \item \(\forall M,\,N \in \mathbf{\Lambda}\ (\llbracket MN \rrbracket_\rho = \llbracket M\rrbracket_\rho \bullet \llbracket N \rrbracket_\rho)\).
%     \item \(\forall x \in V\,\forall M\in \mathbf{\Lambda}\,\forall d\in D\ (\, \llbracket \lambda x.\,M\rrbracket_\rho \bullet d = \llbracket M \rrbracket_{\rho[x/d]})\).
%     \item Dla dowolnych wartościowań \(\rho\),  \(\sigma\) i dowolnego termu \(M\in\nobreak\mathbf{\Lambda}\), jeśli \[\forall x\in\nobreak\mathrm{FV}(M)\ (\rho(x) =\nobreak \sigma(x)),\] to \(\llbracket M\rrbracket_\rho =\nobreak\llbracket M \rrbracket_\sigma\).
%     \item \(\forall M\in\mathbf{\Lambda}\,\forall x, y\in V\ (y\not\in\mathrm{FV}(M)\implies \llbracket[\lambda x.\,M\rrbracket_\rho = \llbracket \lambda y.\,M[x/y]\rrbracket_\sigma) \).
%     \item \(\forall M,\,N\in\mathbf{\Lambda}\,\left(\forall d\in D\,(\llbracket M\rrbracket_{\rho[x/d]}=\llbracket N \rrbracket_{\rho[x/d]} ) \implies \llbracket \lambda x.\,M\rrbracket_\rho = \llbracket \lambda x.\,N\rrbracket_\rho\right)\).
%   \end{enumerate}
% \end{definicja}

\begin{definicja}[Ekstensjonala równoważność]%3.4
Niech \((D,\,\bullet)\) będzie strukturą aplikatywną i niech \(a,\,b\in D\). Określamy relację: 
\begin{align*}
a \sim b \quad \Leftrightarrow\quad \forall d \in D \left(a \bullet d = b \bullet d\right).
\end{align*}
Powiemy, że \(a\) jest \emph{ekstensjonalnie równoważne} \(b\), jeśli \(a\sim b\).
\end{definicja}

\begin{definicja}% kombinacja zmiennycha, 3.7
\begin{enumerate}
  \setlength\itemsep{0em}
\item
Powiemy, że term \(M\) jest \emph{kombinacją zmiennych}, jeśli \(M\equiv x\) dla pewnej zmiennej \(x\) albo \(M\equiv (PQ)\) dla pewnych kombinacji zmiennych \(P\) i \(Q\). 
\item 
Niech \(M\) będzie kombinacją zmiennych taką, że \(\mathrm{FV}(M)=\{x_1,\,x_2,\,\dots,\,x_n\}\). Względem \(M\) określamy funkcję \(f_M: D^n\to D\) następującym wzorem:

% \begin{align*}
% f_M (d_1,\,d_2,\,\dots,\,d_n) = \begin{cases}
% d_k,\ \text{jeśli}\ M\equiv x_k \text{dla pewnego}\ 1\leq k \leq n,\\
% f_Q (d_1,\,d_2,\,\dots,\,d_n) \bullet f_Q (d_1,\,d_2,\,\dots,\,d_n)\  \text{jeśli}\ M\equiv (QR)\ \text{dla pewnych kombinacji zmiennych QR}.
% \end{cases}
% \end{align*}
\begin{align*}
f_M (d_1,\,d_2,\,\dots,\,d_n) = \begin{cases}
d_k,\ &\text{jeśli}\ M\equiv x_k, \\
f_Q (d_1,\,d_2,\,\dots,\,d_n) \bullet f_Q (d_1,\,d_2,\,\dots,\,d_n),\ & \text{jeśli}\ M\equiv (PQ).
\end{cases}
\end{align*}

\item
Powiemy, że struktura aplikatywna \((D,\,\bullet)\) jest \emph{kombinatorycznie zupelna}, jeśli dla każdej kombinacji zmiennych \(M\) istnieje \(a_M\in D\) taki, że dla wszystkich \(d_1\),\(d_2\),\(\dots\),\(d_n\in D\)
\[
  a_M\bullet d_1 \bullet d_2 \bullet \dots \bullet d_n = f_M(d_1,\, d_2,\,\dots,\,d_n).
\]
\end{enumerate}
\end{definicja}

% \begin{twierdzenie}%3.8
% Struktura aplikatywna \((D,\,\bullet)\) jest kombinatorycznie zupełna wtedy i tylko wtedy, gdy istnieją takie \(k,\,s\in D\), że dla wszystkich \(a,\,b,\,c\in D\) mamy: 
% \begin{align*}
% k\bullet a\bullet b = a\quad \text{oraz}\quad s\bullet a \bullet b \bullet c = a\bullet c \bullet (b \bullet c).
% \end{align*}
% \end{twierdzenie}

\begin{definicja}[Model bezsyntaktyczny]%3.9
  Modelem bezsyntaktycznym rachunku \(\lambda\) nazywamy trójkę \((D,\,\bullet, \Lambda)\), gdzie \(\mathbb{D}=(D,\,\bullet)\) jest strukturą aplikatywną, \(\Lambda: D \to D\) i spełnione są poniższe własności:
  \begin{enumerate}[label={(\alph*)}, ref={(\alph*)}]
  \setlength\itemsep{0em}
    \item \(\mathbb{D}\) jest kombinatorycznie zupełny.
    \item \(\Lambda (a) \sim a\) dla wszystkich \(a\in D\).
    \item Jeśli \(a\sim b\), to \(\Lambda(a) = \Lambda(b)\) dla wszystkich \(a,\,b\in D\).
    \item \(\Lambda(a)=e\bullet a\) dla pewnego \(e\in D\) i wszystkich \(a\in D\).
  \end{enumerate}
\end{definicja}

\paragraph{Elementy teorii porządku} 
Niech \((D,\sqsubseteq)\) będzie zbiorem częściowo uporzadkowanym. Powiemy, że \(b\in D\) jest elementem \emph{najmniejszym}, jesli \(b\sqsubseteq d\) dla wszystkich \(d\in D\). Element ten, o ile istnieje, wyznaczony jest jednoznacznie i będziemy oznaczać go symbolem \(\perp\). Niech \(X\subset D\). \emph{Ograniczeniem górnym} \(X\) nazywamy element \(u\in D\) taki, że \(x\sqsubseteq u\) dla wszystkich \(x\in X\). \emph{Kres górny} zbioru \(X\) nazywamy element \(\ell\in D\) taki, że \(\ell\) jest ograniczeniem górnym \(X\) i \(\ell\sqsubseteq u\) dla wszystkich ograniczeń górnych \(u\) zbioru \(X\). Element taki, o ile istnieje, będziemy oznaczali symbolem \(\bigsqcup X\). Podzbiór \(X\subset D\) nazywamy \emph{skierowanym}, jeśli \(X\neq\emptyset\) i dla dowolnych \(x, y\in X\) istnieje \(z\in X\) taki, że \(x\sqsubseteq z\) i \(y\sqsubseteq z\). 

\begin{definicja}[Zupełny porządek częściowy]%5.1
Porządek częściowy \((D,\,\sqsubseteq)\) taki, że
\begin{enumerate}[label={(\alph*)}, ref={(\alph*)}]
  \setlength\itemsep{0em}
  \item posiada element najmniejszy oraz
  \item każdy skierowany poddzbiór \(X\subset D\) ma kres górny,
\end{enumerate}
  nazywamy \emph{zupełnym porządkiem częściowym} (w skrócie: \emph{cpo}).
\end{definicja}

Ustalmy, że jeśli \(D'\), \(D''\), \(\dots\)  są cpo, to odpowiadające im porządki częściowe będziemy notowali symbolami \(\sqsubseteq'\), \(\sqsubseteq''\), \(\dots\)

\begin{przyklad}
  Ustalmy \(\perp\not\in \mathbb{N}\) i niech \(\mathbb{N}^{+}=\mathbb{N}\cup{\perp}\). Określmy na \(\mathbb{N}^+\) następującą relację:
  \begin{align*}
    a \sqsubseteq b \quad \Leftrightarrow\quad (a=\perp\ \land\ b\in \mathbb{N})\ \lor\ a = b
  \end{align*}
  \(\sqsubseteq\) jest oczywiście zwrotna, przechodnia i antysymetryczna. Widzimy, że \(\mathbb{N}^{+}\) ma względem niej element najmniejszy (jest nim  \(\perp\)) i oczywiście każdy podzbiór \(\mathbb{N}^{+}\) jest skierowany. 
\end{przyklad}

\begin{definicja}[Monotoniczność, ciągłość]\label{def:m_cont}%5.3
Niech \(D\) i \(D'\) bedą cpo i \(\varphi: D\to D'\).l
\begin{enumerate}[label={(\alph*)}, ref={(\alph*)}] 
  \setlength\itemsep{0em}
\item Powiemy, że \(\varphi\) jest \emph{monotoniczna}, jeśli \(\varphi(a) \sqsubseteq' \varphi(b)\) dla \(a\sqsubseteq b\).
\item Powiemy, że \(\varphi\) jest \emph{ciągła (w sensie Scotta}\footnote{Funkcje te są ciągłe w topologicznym sensie względem topologii Scotta.}), jeśli \(\varphi(\bigsqcup X) = \bigsqcup \varphi (X)\) dla wszystkich skierowanych podzbioru \(X\subseteq D\).\label{def:m_cont_2}
\end{enumerate}
Symbolem \([D\to D']\) oznaczamy zbiór wszystkich funkcji ciągłych ze zbioru \(D\)~do~\(D'\).
\end{definicja}

Zauważmy, że jeśli \(\varphi\) jest ciągła, to jest również monotoniczna. Istotnie, jeśli \(a\sqsubseteq b\), to w szczególności \(\{a, b\}\subseteq D\) jest skierowanym podzbiorem \(D\). Wówczas \(\bigsqcup\{a, b\}=b\) i ponieważ \(\bigsqcup \varphi(\{a\,b\})=\varphi(\bigsqcup\{a,\,b\})\), to otrzymujemy, że \(\varphi(a)\sqsubseteq \varphi(\bigsqcup\{a,\,b\}) = \varphi(b)\).


\begin{twierdzenie}%5.4
Jeśli \(D\) i \(D'\) są cpo, to \([D\to D']\) jest cpo.
  % \begin{enumerate}[label={(\roman*)}, ref={(\roman*)}]
  % \setlength\itemsep{0em}
    % \item \(([D\to D'], \preceq)\) jest cpo.
    % \item \(([D\to D'], \preceq)\) ma element najmniejszy.
    % \item Każdy skierowany podzbiór \(\Phi \in [D\to D']\) ma supremum.
     % \item \([D\to D']\) jest cpo.
     % \item \([D\to D']\) ma element najmniejszy.
     % \item Każdy skierowany podzbiór \(\Phi \in [D\to D']\) ma kres górny.
  % \end{enumerate} 
\end{twierdzenie}
\begin{dowod}
  % \begin{enumerate}[label={(\roman*)}, ref={(\roman*)}]
  % \setlength\itemsep{0em}
  %   \item 
      Określmy na \([D\to D']\) relację  \(\preceq\):
\[
\varphi \preceq \psi\quad \Leftrightarrow\quad \forall d\in D \left(\varphi(d) \sqsubseteq' \psi(d)\right),
\]
Ponieważ \(D'\) jest cpo, to aby wykazać, że \([D\to D']\) ma element najmniejszy, wystarczy, że rozpatrzymy \(\perp(d)=\perp'\) dla \(d\in D\).

Niech teraz \(\Phi\) będzie skierowanym podzbiorem \([D\to D']\). 
Wówczas  dla  wszystkich  \(d\in D\)  zbiór  \(\{\varphi(d)\  |\ 
      \varphi\in\Phi\}\) jest  skierowanym podzbiorem  \(D'\). 

      Istotnie, wybierzmy \(y_1,\,y_2\in\{\varphi(d)\ |\ \varphi\in\Phi\}\). Wówczas dla pewnych \(\varphi_1, \varphi_2\in\Phi\) mamy, że \(y_1=\varphi_1(d)\) oraz \(y_2=\varphi_2(d)\). Ponieważ \(\Phi\) jest skierowany, to istnieje \(\varphi_3\) taki, że \(\varphi_1\sqsubseteq \varphi_3\) i \(\varphi_2\sqsubseteq \varphi_3\). Z monotonicznośc funkcji ciągłych i faktu, że \(d\sqsubseteq d\) widzimy, że \(\{\varphi(d)\  |\ \varphi\in\Phi\}\) jest skierowanym podzbiorem \(D'\).

      Określmy funkcję \(\Psi_\Phi: D\to D'\) następującym wzorem:
      \begin{align*}
        \Psi_\Phi(d) = \bigsqcup \Phi d,
      \end{align*}
      gdzie \(\Phi d=\{\varphi(d)\ |\ \varphi \in \Phi\}\). Wystarczy pokazać, że \(\Psi_\Phi\) jest ciągla i jest kresem górnym zbioru \(\Phi\).
      \begin{enumerate}
        \item Pokażemy najpierw, że \(\Psi_\Phi\) jest ciągła. Niech \(X\) będzie dowolnym skierowanym podzbiorem \(D\). Wówczas:
          \begin{align*}
  \Psi_\Phi(\bigsqcup X)&= \bigsqcup \Phi (\bigsqcup X)\\
                        &= \bigsqcup \{\varphi(\bigsqcup X)\ |\ \varphi\in\Phi\}\\
                        &= \bigsqcup \left\{\bigsqcup\left\{\varphi(d)\ |\ d\in X\right\}\ |\ \varphi\in \Phi\right\}\\
                        &= \bigsqcup \left\{\bigcup\limits_{\varphi\in\Phi}\{\varphi(d)\ |\ d\in X\}\right\}\\
                        &= \bigsqcup \left\{\bigcup\limits_{\varphi\in\Phi}\bigcup\limits_{d\in X}\{\varphi(d)\}\right\}
                         =\bigsqcup \left\{\bigcup\limits_{d\in X}\bigcup\limits_{\varphi\in\Phi}\{\varphi(d)\}\right\}\\
                        &= \bigsqcup \left\{\bigcup\limits_{d\in X}\{\varphi(d)\ |\ \varphi\in\Phi \}\right\}\\
                        &= \bigsqcup \left\{\bigsqcup \{\varphi(d)\ |\ \varphi\in\Phi \}\ |\ d\in X\right\}\\
                        &= \bigsqcup \left\{\bigsqcup \Phi d\ |\ d\in X\right\}
                        = \bigsqcup \Psi_\Phi(X).
          \end{align*}
        \item 
      \end{enumerate}
      \qed
\end{dowod}

\begin{definicja}[Projekcja]%5.5
Niech \(D\) i \(D'\) będą cpo. \emph{Projekcją} z \(D'\) do \(D\) nazywamy parę \((\varphi, \psi)\) funkcji \(\varphi \in [D\to D']\), \(\psi \in [D'\to D]\) takich, że
\begin{align}
\psi\circ \varphi = I_D\quad \text{oraz}\quad \varphi\circ \psi \preceq I_{D'} \tag{\textasteriskcentered},
\end{align}
gdzie przez \(I_D\) i \(I_{D'}\) oznaczamy funkcję identycznościową na zbiorze \(D\) i \(D\), odpowiednio.
\end{definicja}

\begin{definicja}%5.6
Dla \(m,\,n\geq 0\) określamy \(\varphi_{mn}:D_m \to D_n\) w następujący sposób:
\begin{align*}
\varphi_{mn} =
\begin{cases}
\varphi_{n-1} \circ \varphi_{n-2} \circ \dots \varphi_{m+1} \circ \varphi_m, & \text{jeśli}\ m\leq n,\\
I_{D_n}, & \text{jeśli}\ m=n,\\
\psi_n \circ \psi_{n+1} \circ \dots \circ \psi_{m-2}\circ \psi_{m-1} & \text{jeśli}\ m\geq n.
\end{cases}
\end{align*}
\end{definicja}

\begin{fakt}[{\cite[Lemat 16.33]{Hindley:2008:LCI:1388400}}]%5.7
Niech \(m,\,n\geq 0\). Wówczas
\begin{enumerate}[label={(\roman*)}, ref={(\roman*)}] 
  \setlength\itemsep{0em}
\item \(\varphi_{mn}\in [D_m\to D_n]\),
\item jeśli \(m\leq n\), to \(\varphi_{nm}\circ \varphi_{mn} = I_{D_m}\),
\item jeśli \(m>n\), to \(\varphi_{nm}\circ \varphi_{mn} \preceq I_{D_m}\),
\item jeśli \(m<n\), to \((\varphi_{mn},\varphi_{nm})\) jest projekcją z \(D_n\) do \(D_m\),
\item jeśli \(m<k<n\) lub \(n<k<m\), to \(\varphi_{kn}\circ\varphi{mk}=\varphi_{mn}\).
\end{enumerate}
\end{fakt}

\paragraph{Konstrukcja \(D_\infty\)}

\begin{definicja}%5.8
Niech \(D_\infty\) oznacza zbiór wszystkich nieskończonych ciągów postaci
\[
d=(d_0,\,d_1,\,\dots)
\]
takich, że dla wszystkich \(n\geq 0\) mamy, że \(d_n\in D_n\) oraz \(\psi_n (d_{n+1}) = d_n\). Przez \(d_n\) oznaczać będziemy \(n\)-ty element ciągu \(d\). Jeśli \(X\subset D_\infty\), to okreslamy \(X_n=\{ d_n\ |\ d\in X\).

Na \(D_\infty\) określamy relację \(\sqsubseteq\) w nastepujący sposób:
\[
(d_0,\,d_1,\,\dots) \sqsubseteq (d_0',\,d_1',\,\dots) \quad \Leftrightarrow\quad  \forall n\geq 0\  (d_n\sqsubseteq d_n') 
\]
\end{definicja}

\begin{fakt}[{\cite[Lemat 16.36]{Hindley:2008:LCI:1388400}}]%5.10
\begin{enumerate}[label={(\roman*)}, ref={(\roman*)}] 
  \setlength\itemsep{0em}
\item \(D_\infty\) jest cpo.
\item \(D_\infty\) zawiera element najmniejszy \(\perp=(\perp_0,\,\perp_1,\,\dots)\), gdzie przez \(\perp_n\) oznaczamy najmniejszy element \(D_n\).
\item Kres górny każdego skierowany zbioru \(X\subset D_\infty\) ma postać
\[
\bigsqcup X = (\bigsqcup X_0,\,\bigsqcup X_1,\,\dots)
\]
\end{enumerate}
\end{fakt}

\begin{definicja}%5.11
Dla \(n\geq 0\) okreslamy funkcje \(\varphi_{n\infty}:D_n\to D_\infty\) oraz \(\varphi_{\infty n}: D_\infty \to D_n\), gdzie 
\begin{align*}
\varphi_{n\infty}(d) &= (\varphi_{n0}(d),\,\varphi_{n1}(d),\,\dots)\ \text{dla}\ d\in D_n,\\
\varphi_{\infty n}(d) &= d_n.
\end{align*}
\end{definicja}

\begin{fakt}[{\cite[Lematy 16.38, 16.39, 16.42]{Hindley:2008:LCI:1388400}}]%5.12
Niech \(m,\,n\geq 0\), \(m\leq n\) i \(a,\,b\in D_\infty\). Wówczas:
\begin{enumerate}[label={(\roman*)}, ref={(\roman*)}] 
  \setlength\itemsep{0em}
\item \((\varphi_{n\infty},\,\varphi_{\infty n})\) jest projekcją z \(D_\infty\) do \(D_n\),
\item \(\varphi_{mn}(a_m)\sqsubseteq a_n\),
\item \(\varphi_{m\infty}(a_m)\sqsubseteq \varphi_{n\infty}(a_n)\),
\item \(a=\bigsqcup_{n\geq 0}\varphi_{n\infty}(a_n),\),
\item \(\varphi_{n\infty}(a_{n+1}(b_n))\sqsubseteq\varphi_{(n+1)\infty}(a_{n+2}(b_{n+1}))\).
\end{enumerate}
\end{fakt}

\begin{definicja}%5.13
  Dla \(a,\,b\in D_\infty\) określamy:
\begin{align*}
a \bullet b = \bigsqcup\left\{\varphi_{n\infty}(a_{n+1}(b_n))\ |\ n\geq 0\right\}\tag{\textasteriskcentered}
\end{align*}
\end{definicja}

% \begin{definicja}%5.14
%   \begin{enumerate}
%     \item 
%   \end{enumerate}
% \end{definicja}
