\subsection{Podsumowanie}


\begin{minted}{haskell}
module Fnord where
  comp = \g f x -> g (f x)
\end{minted}

\begin{minted}{haskell}
comp
  :: forall t_aqz t1_aqB t2_aqD.
     (t_aqz -> t1_aqB) -> (t2_aqD -> t_aqz) -> t2_aqD -> t1_aqB

comp =
  \ (@ t_aqz)
    (@ t1_aqB)
    (@ t2_aqD)
    (g_aqg :: t_aqz -> t1_aqB)
    (f_aqh :: t2_aqD -> t_aqz)
    (x_aqi :: t2_aqD) ->
    g_aqg (f_aqh x_aqi)
\end{minted}


Na kanwie zaproponowanej przez H. P. Barendregta w \cite[Rozdział 5]{Barendregt_1992} klasyfikacji rozszerzeń rachunku \(\lambda\) z typami prostymi (tzw. \emph{kostki \(\lambda\)}, Rysunek \ref{fig:lambda-cube}), rozdział ten poświęcimy omówieniu wzajemnych zależności jakie mogą łączyć \(\lambda\)-termy i typy. Zajmować będziemy się wyłącznie systemami w stylu Churcha.

\begin{figure}[!h]
  \centering
  \begin{tikzpicture}
  \matrix (m) [matrix of math nodes,
  row sep=3em, column sep=3em,
  text height=1.5ex,
  text depth=0.25ex]{
              & \lambda\omega             &              & \lambda P \omega             \\
  \lambda 2   &                           & \lambda P 2                                \\
              & \lambda\underline{\omega} &              & \lambda P \underline{\omega} \\
  \lambda{\to}&                           & \lambda P  \\
  };
  \path[-{Latex[length=2.5mm, width=1.5mm]}]
  (m-1-2) edge (m-1-4)
  (m-2-1) edge (m-2-3)
          edge (m-1-2)
  (m-3-2) edge (m-1-2)
          edge (m-3-4)
  (m-4-1) edge (m-2-1)
          edge (m-3-2)
          edge (m-4-3)
  (m-3-4) edge (m-1-4)
  (m-2-3) edge (m-1-4)
  (m-4-3) edge (m-3-4)
          edge (m-2-3);
  \end{tikzpicture}
  \caption{Poszczególne systemy klasyfikacji H. Barendregta; kierunek krawędzi \(\to\)  oznacza relację \(\subseteq\).}\label{fig:lambda-cube}
\end{figure}


