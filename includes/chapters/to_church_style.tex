\subsection{Typy w stylu Churcha}\label{subsec:church_style}
Przypisanie typu \(\lambda\)-termowi rozpoczynamy zawsze od określenia typów dla \(\lambda\)-zmiennych. Zasadniczo możemy to rozwiązać na dwa sposoby:

\begin{enumerate}
  \item Możemy przypisać unikalny typ każdej \(\lambda\)-zmiennej przed jej wprowadzeniem. Takie podejście nazywamy \emph{stylem Churcha} albo typowaniem \emph{explicite}, ponieważ deklaracje typowe zmiennych występują jawnie w składni \(\lambda\)-termów. W konsekwencji w podejściu tym nie spotykamy problemu typowalności. Stąd systemy w tym stylu nazywa się również systemami {typowanymi} (ang. \emph{typed systems}).
  \item Inny sposób polega na nie ustalaniu typów zmiennych. Składnia \(\lambda\)-termów nie ulega wówczas zmianie, zaś o typie rozstrzyga algorytm rekonstrukcji typu. Typy w tym stylu były przedmotem Rozdziału \ref{sec:simple_types}. W literaturze powszechnie nazywa się to podejście \emph{stylem Currego} albo typowaniem \emph{implicite}, zaś systemy w takim stylu określa się systemami \emph{przypisywania typu} (ang. \emph{type assignment systems}).
\end{enumerate}

Obydwa podejścia dają w rezultacie takie same systemy typów \cite[Rozdział 3.4]{Urzyczyn2006}.  % Systemy w stylu Currego odpowiadają językom programowania funkcyjnego takim jak ML, Clean czy Haskell. Programując w takich językach rolę wyprowadzania typu w istocie przejmuje kompilator, zaś deklaracje typów mają tylko charakter dokumentacji.  
Rozwiązaniem kompromisowym jest tzw. typowanie w \emph{stylu de Brujina}\footnote{W \cite{Urzyczyn2006} nazywa się to podejście \emph{nieortodoksyjnym stylem Churcha}} \cite[1A.33]{barendregt_dekkers_statman_2013} w którym nie ustala się typu wszystkich zmiennych, jednak adnotacje typowe są częścią składni (tak jak w stylu Churcha) i zależą od ustalonego kontekstu. 

Zaprezentujemy teraz alternatywną składnię oraz reguły wyprowadzania typów dla systemu typów prostych w stylu Churcha. Wszystkie określenia oraz twierdzenia występujące dotychczas w Rozdziale \ref{sec:simple_types} mają swoje odpowiedniki dla systemu w stylu Churcha \cite[Rozdział 2.10]{nederpelt_geuvers_2014}. Wyjątek stanowi Twierdzenie \ref{thm:polymorphism}, które jest zastąpione w tym systemie Twierdzeniem \ref{thm:type_uniqueness} o jednoznaczności typu.

\subsubsection{Składnia}
Zbiór typów \(\mathbb{T}\) definiujemy w myśl Definicji \ref{def:typy-proste}. Niech \(U,\,V\) będą przeliczalnie nieskończonymi zbiorem zmiennych przedmiotowych, odpowiednio: zmiennych typowych i (\(\lambda\)-zmiennych). Celem zdefiniowania \(\lambda\)-termów w stylu Churcha przeprowadzamy konstrukcję analogiczną do tej przedstawionej w Rozdziale \ref{sec:untyped_lambda}: okreslamy zbiór pretermów \(\mathbb{\tilde\Lambda}_\mathbb{T}\), a następnie definiujemy \(\lambda\)-termy jako klasy abstrakcji \(\alpha\)-konwersji.
\begin{align*}
  \mathbb{T}\ &\leftarrow \ U\ |\ (\mathbb{T}\to\mathbb{T})\\
  \mathbf{\tilde\Lambda}_\mathbb{T}\ &\leftarrow \ V\ |\ (\mathbf{\tilde\Lambda}_\mathbb{T}\,\mathbf{\tilde\Lambda}_\mathbb{T}) \ |\ (\lambda V:\mathbb{T}.\, \mathbf{\tilde\Lambda}_\mathbb{T})
\end{align*}

Zauważmy, że \(\lambda\)-termy w stylu Churcha różnią się od stylu Currego tylko w wypadku \(\lambda\)-abstrakcji.  Z tą jedną modyfikacją definicje zbioru zmiennych wolnych, podstawienia, \(\beta\)- i \(\eta\)-redukcji, kontekstu i wyprowadzenia są analogiczne do tych z Rozdziałów \ref{sec:untyped_lambda} i \ref{sec:simple_types}.

\subsubsection{Typowanie}

\begin{center}
    \begin{tabular}{ ccc}
      \(\Gamma \vdash x:\sigma\ \text{(var)}\), & jeśli \(x:\sigma\in\Gamma\),
      \vspace{1em}\\
      {\begin{prooftree}
        \Hypo{\Gamma \vdash M:\sigma \to \tau} \Hypo{ \Gamma \vdash N:\sigma}
        \Infer2[(app)]{\Gamma \vdash (MN):\tau}
      \end{prooftree}},
      \vspace{1em}
      \\ 
      {\begin{prooftree}
        \Hypo{ \Gamma, x:\sigma \vdash M:\tau}
        \Infer1[(abs)]{\Gamma \vdash (\lambda\, x:\sigma.\, M):\sigma\to\tau}
      \end{prooftree}}.
      \end{tabular}
\end{center}

Zauważmy, że mając zadany kontekst, typ każdego poprawnego \(\lambda\)-termu jest jednoznacznie określony. Jest to istotna różnica, którą wprowadza styl Churcha. W~systemach w stylu Currego termy poprawne są zamknięte ze względu na podstawienie typu. Własność, którą wyraża Twierdzenie \ref{thm:type_uniqueness} zachodzi w nich z dokładnością do podstawienia. 

\begin{twierdzenie}\label{thm:type_uniqueness}(O jednoznaczności)
  Jeśli \(\Gamma\vdash M:\sigma\) i \(M\to^{*}_\beta\), to \(\Gamma\vdash N:\sigma\).
\end{twierdzenie}
\begin{dowod}
  Dowód przebiega przez indukcję względem długości wyprowadzenia \(\Gamma\vdash M:\sigma\). Szczegóły pomijamy.\qed
\end{dowod}

\begin{przyklad}
%    \begin{enumerate}[label=(\alph*)]
%    \item
Zauważmy, że nie istnieje jeden typ dla reprezentacji funkcji identycznościowej. Jeśli \(nat\) jest stałą typową, którą reprezentujemy liczby naturalne, to identyczność na zbiorze liczb naturalnych będziemy reprezentowali termem \(\lambda x:nat.\,x\), na zbiorze funkcji \(\mathbb{N}\to\mathbb{N}\), \(\lambda x:\mathrm{nat}\to\mathrm{nat}.\,x\) i tak dalej.
  Aby okreslić ogólną postać identyczności, musimy móc abstrahować po zbiorze typów, czyli parametryzować postać termu typem. %Wymaga to dodania kolejnej abstrakcji:
      \begin{align*}
        \lambda \sigma:* .\,\lambda x:\sigma.\,x,
      \end{align*}
      gdzie symbolem \(*\) oznaczamy typ obiektów będących typami (szczegóły omówimy w Rozdziale \ref{sec:system_f}).
      Własność tę (polimorfizm parametryczny) miał w pewnym sensie rachunek \(\lambda\) w stylu Currego (Podrozdział \ref{subsec:polymorphism}). 
%    \item
%      Przykład dotyczy iteracji. Przypuśćmy, że term \(F\) jest typu \(\sigma\to\sigma\) i okreslamy \(D_{\sigma F}\equiv\lambda x:\nobreak\sigma.\,F(Fx)\). Chcąc określić \(D\) dla dowolnych \(F\) i \(\sigma\)
%  \end{enumerate}
\end{przyklad}

