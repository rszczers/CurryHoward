\subsection{Równościowa teoria rachunku \(\lambda\)}\label{subsec:lambda-formal-theory}
Rachunek \(\lambda\) możemy rozszerzyć o teorię równościową w stylu Hilberta. Składa się ona z następujących reguł i aksjomatów inferencji.
\begin{enumerate}[label={(\alph*)}, ref={(\alph*)}]
\setlength\itemsep{0em}
\item Aksjomaty:
\begin{enumerate}[label={(\greek*)}, ref={(\greek*)}]
  \setlength\itemsep{0em}
  \item \(\lambda x.\,M = \lambda y.\,M[x/y]\), jeśli \(y\in\mathrm{FV}(M)\)
  \item \((\lambda x.\,M)N = M[x/N]\)
  \item[(\(\rho\))]{ \(M=M\)}
\end{enumerate}

\item Reguły inferencji:
\begin{center}
{\setlength{\extrarowheight}{20pt}%
\begin{tabular}{cc}
  {\begin{prooftree}
    \Hypo{M=M'}
    \Infer1[(l-con)]{NM=NM'}
  \end{prooftree}}
  &
  {\begin{prooftree}
    \Hypo{M=N} \Hypo{N=P}
    \Infer2[(trans)]{M=P}
  \end{prooftree}}
  \\
  {\begin{prooftree}
    \Hypo{M=M'}
    \Infer1[(r-con)]{MN=M'N}
  \end{prooftree}}
  &
  {\begin{prooftree}
    \Hypo{M=N}
    \Infer1[(sym)]{N=M}
  \end{prooftree}}
  \\
  {\begin{prooftree}
    \Hypo{M=M'}
    \Infer1[(\(\xi\))]{\lambda x.\,M=\lambda x.\,M'}
  \end{prooftree}}
  \end{tabular}}
\end{center}
\end{enumerate}

Powyższy system nazywamy \emph{teorią \(\lambda\beta\)}; jest ona zbiorem \emph{równości} między \(\lambda\)-termami. 

\begin{definicja}(Wyprowadzenie)
Niech \(M, N\in\mathbf{\Lambda}\) będą dowolnymi \(\lambda\)-termami.
\emph{Wyprowadzeniem} równości \(M=N\) ze zbioru równości \(\Gamma\) w teorii \(\lambda\beta\) będziemy nazywali drzewo \(\mathcal{P}\) równości takie, że:
\begin{enumerate}[label={(\roman*)}, ref={(\roman*)}]
  \setlength\itemsep{0em}
  \item liście \(\mathcal{P}\) są aksjomatami albo należą do zbioru \(\Gamma\),
  \item korzeniem \(\mathcal{P}\) jest równość \(M=N\) (jest \emph{wnioskiem}),
  \item wszystkie pozostałe równości w \(\mathcal{P}\) są uzyskane ze swoich dzieci (\emph{przesłanek}) za pomocą reguł inferencji.
\end{enumerate}

Jeśli istnieje wyprowadzenie \(\mathcal{P}\) równości \(M=N\) ze zbioru równości \(\Gamma\), to \(M=N\) nazywamy \emph{wyprowadzalnym w kontekście} \(\Gamma\) i piszemy \(\lambda,\,\Gamma\vdash M=N\). Jeśli \(\Gamma\) jest zbiorem pustym, to wyprowadzenie \(\mathcal{P}\) nazywamy \emph{dowodem} równości \(M=N\), a o równości \(M=N\) mówimy, że jest \emph{dowodliwa}.
\end{definicja}

Następujący fakt ustala związek między równościami dowodliwymi w \(\lambda\beta\) a \(\lambda\)-termami, które są swoimi \(\beta\)-konwersami.
\begin{fakt}(\cite[Lem. 6.4]{Hindley:2008:LCI:1388400})
Niech \(M, N\in\mathbf{\Lambda}\). Wówczas
\[
M=_\beta N \iff \lambda \vdash M=N
\]
\end{fakt}
