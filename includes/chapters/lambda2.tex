\subsection{System \(\lambda 2\) (System F)}\label{subsec:lambda2}
System \(\lambda  2\) wprowadzony został przez J.-Y.  Girarda jako System
F i w literaturze szerzej znany jest pod tą nazwą. 

\begin{definicja}(Typów \(\mathbb{T}2\))
  Niech \(\mathbb{V}\) będzie przeliczalnie nieskończonym zbiorem zmiennych przedmiotowych. Zmienne te będziemy nazywali \emph{zmiennymi typowymi} i oznaczali literami alfabetu greckiego (\(\alpha, \beta, \gamma, \dots\)). Zbiór typów \(\mathbb{T}2\) systemu \(\lambda 2\) okreslamy w notacji BNF następującym zapisem:
  \begin{align*}
    \mathbb{T2}\ &\leftarrow\ \mathbb{V}\ |\ (\mathbb{T2}\to\mathbb{T2})\ |\ (\Pi \mathbb{V}:*.\,\mathbb{T2})
  \end{align*}
\end{definicja}
\begin{definicja}(Pretermy \(\mathbf{\tilde\Lambda}_\mathbb{T2}\))
  Niech \(V\) będzie przeliczalnie nieskończonym zbiorem zmiennych przedmiotowych. Zmienne te będziemy nazywali \emph{zmiennymi termowymi} i oznaczali literami alfabetu łacińskiego (\(x,\, y,\, z,\,\dots\)). Zbiór pretermów \(\mathbf{\tilde\Lambda}_\mathbb{T2}\) systemu \(\lambda 2\) okreslamy w notacji BNF następującym zapisem:
  \begin{align*}
      \mathbf{\tilde\Lambda}_\mathbb{T2}\ &\leftarrow \ V\ |\ (\mathbf{\tilde\Lambda}_\mathbb{T2}\,\mathbf{\tilde\Lambda}_\mathbb{T2}) \ |\ (\mathbf{\tilde\Lambda}_\mathbb{T2}\,\mathbb{T2}) \ |\ (\lambda V:\mathbb{T2}.\, \mathbf{\tilde\Lambda}_\mathbb{T2})\ |\ (\lambda V:*.\, \mathbf{\tilde\Lambda}_\mathbb{T2})
  \end{align*}
  Wyrażenia postaci \((\lambda V:*.\, \mathbf{\tilde\Lambda}_\mathbb{T2})\) i \((\mathbf{\tilde\Lambda}_\mathbb{T2}\,\mathbb{T2})\) nazywamy \emph{polimorficzną abstrakcją} i \emph{polimorficzną aplikacją}, odpowiednio. Stosujemy standardowe konwencje notacyjne. O zmiennej (termowej lub typowej) występującej bezpośrednio po znaku \(\lambda\) powiemy, że jest związana. 
\end{definicja}

Wyrażenia \(\lambda\) (\(\lambda\)-termy) w systemie \(\lambda 2\) to klasy abstrakcji \(\alpha\)-konwersji. Konstrukcja jest analogiczna do zaprezentowanej w Rozdziale \ref{subsec:lambda_terms_untyped}, ale aby nie wchodzić w szczegóły, ustalmy, że utożsamiać będziemy wyrażenia, które różnią się między sobią wyłącznie zmiennymi związanymi.

\begin{definicja}(Stwierdzenie, deklaracja)
  Odpowiedniej modyfikacji ulegają pojęcia wprowadzone w Definicji \ref{def:statement_simple}.
  \begin{enumerate}
  \setlength\itemsep{0em}
    \item Stwierdzeniem nazywamy każdy napis postaci \(M:\sigma\), gdzie \(M\in\Lambda_\mathbb{T2}\) i \(\sigma\in \mathbb{T2}\) lub \(\sigma:*\), gdzie \(\sigma\in\mathbb{T2}\).
    \item  Deklaracją nazywamy każde stwierdzenie ze zmienna typowa lub zmienna termowa w miejscu podmiotu.
  \end{enumerate}
\end{definicja}
\begin{definicja}(\(\lambda 2\)-kontekst, dziedzina, zakres)
  \begin{enumerate}
  \setlength\itemsep{0em}
  \item \(\emptyset\) jest \(\lambda 2\)-kontekstem; oznaczamy go parą nawiasów 
    \(()\).
  \item Jeśli:
    \begin{enumerate}
    \item \(\Gamma\) jest \(\lambda 2\)-kontekstem,
    \item \(\alpha\in\mathbb{V}\) jest zmienną typową taką, że \(\alpha\not\in \mathrm{dom}(\Gamma)\),
    \end{enumerate}
    to \(\Gamma, \alpha:*\) jest \(\lambda 2\)-kontekstem, gdzie 
      \begin{align*} 
        \mathrm{dom}(\Gamma, \alpha:*)&=(\mathrm{dom}(\Gamma),\, \alpha),\\
        \mathrm{rg}(\Gamma, \alpha:*)&=\mathrm{rg}(\Gamma)\cup\{*\}.
      \end{align*}
  \item Jeśli:
    \begin{enumerate}
     \setlength\itemsep{0em}
     \item \(\Gamma\) jest \(\lambda 2\)-kontekstem,
     \item \(\rho\in\mathbb{T}2\) jest typem takim, że \(\alpha\in\mathrm{dom}(\Gamma)\) dla wszystkich \(\alpha\in\mathrm{FV}(\rho)\),
     \item \(x\in V\) jest zmienną termową taką, że \(x\not\in\mathrm{dom}\,\Gamma\),
    \end{enumerate}
      to \(\Gamma, x:\rho\) jest \(\lambda 2\)-kontekstem, gdzie 
      \begin{align*} 
        \mathrm{dom}(\Gamma, x:\rho)&=(\mathrm{dom}(\Gamma), x),\\
      \mathrm{rg}(\Gamma, \alpha:*)&=\mathrm{rg}(\Gamma)\cup\{\rho\}.
      \end{align*}
      \(\mathrm{dom}(\Gamma)\) i \(\mathrm{rg}(\Gamma)\) nazywamy odpowiednio \emph{dziedziną} i \emph{zakresem} \(\lambda 2\)-kontekstu \(\Gamma\).
  \end{enumerate}
\end{definicja}
  \subsubsection{Typowanie}
  \begin{definicja}(Kontekst)
  \end{definicja}
  Typowanie: \(\Gamma \vdash M : A\)\\
  \begin{center}
  \begin{tabular}{r c c}

    \vspace{0.5cm}
    (var) &
      \(\Gamma \vdash x:\sigma\), & jesli \(x:\sigma\in\Gamma\)\\
    \vspace{0.5cm}

    (app) &
    {\begin{prooftree}
      \Hypo{\Gamma \vdash M:\sigma \to \tau} \Hypo{ \Gamma \vdash N:\sigma}
      \Infer2[]{\Gamma \vdash MN:\tau}
    \end{prooftree}} & \\
    \vspace{0.5cm}

    (abs) &
    {\begin{prooftree}
      \Hypo{ \Gamma, x:\sigma \vdash M:\tau }
      \Infer1[]{\Gamma \vdash (\lambda\, x:\sigma.\, M):\sigma\to \tau}
    \end{prooftree}} & \\
    \vspace{0.5cm}

    (form) &
      \(\Gamma\vdash B:*\), & jeśli \(B\in\mathbb{T}2\) i  \(\mathrm{FV}(B)\subseteq \mathrm{rg}\,\Gamma\)\\
    \vspace{0.5cm}

    (\(\Pi\)e) &
    {\begin{prooftree}
      \Hypo{ \Gamma \vdash M:(\Pi \alpha:*.\,A)}
      \Hypo{ \Gamma \vdash B:* }
      \Infer2[]{\Gamma \vdash M B : A[\alpha/B]}
    \end{prooftree}} &
     \\
    \vspace{0.5cm}

    (\(\Pi\)i) &
    {\begin{prooftree}
      \Hypo{\Gamma, \alpha : * \vdash M:A } 
      \Infer1[]{\Gamma \vdash \lambda \alpha:*.\,M:\Pi \alpha:*.\,A}
    \end{prooftree}} & \\
  \end{tabular}
  \end{center}

  \begin{definicja}(Poprawność, typowalność)
    Powiemy, że term \(M\in\mathbf{\Lambda}_\mathbb{T2}\) jest \emph{poprawny} lub \emph{typowalny}, jeśli istnieje \(\lambda 2\)-kontekst \(\Gamma\) i typ \(\rho\in T2\) taki, że \(\Gamma\vdash M:\rho\).
  \end{definicja}
  \subsubsection{Redukcja}
  \begin{definicja}(\(\beta\)-redukcja)
  \end{definicja}




