\subsection{System \(\lambda 2\) (System F)}\label{subsec:lambda2}
System \(\lambda  2\), wprowadzony przez J.-Y.  Girarda jako System
F i w literaturze szerzej znany pod tą nazwą. 

\begin{definicja}(Typów \(\mathbb{T}2\))
  Niech \(\mathbb{V}\) będzie przeliczalnie nieskończonym zbiorem zmiennych przedmiotowych. Zmienne te będziemy nazywali \emph{zmiennymi typowymi} i oznaczali literami alfabetu greckiego (\(\alpha, \beta, \gamma, \dots\)). Zbiór typów \(\mathbb{T}2\) systemu \(\lambda 2\) okreslamy w notacji BNF następującym zapisem:
  \begin{align*}
    \mathbb{T2}\ &\leftarrow\ \mathbb{V}\ |\ (\mathbb{T2}\to\mathbb{T2})\ |\ (\Pi \mathbb{V}:*.\,\mathbb{T2})
  \end{align*}
\end{definicja}
\begin{definicja}(Pretermy \(\mathbf{\tilde\Lambda}_\mathbb{T2}\))
  Niech \(V\) będzie przeliczalnie nieskończonym zbiorem zmiennych przedmiotowych. Zmienne te będziemy nazywali \emph{zmiennymi termowymi} i oznaczali literami alfabetu łacińskiego (\(x,\, y,\, z,\,\dots\)). Zbiór pretermów \(\mathbf{\tilde\Lambda}_\mathbb{T2}\) systemu \(\lambda 2\) okreslamy w notacji BNF następującym zapisem:
  \begin{align*}
      \mathbf{\tilde\Lambda}_\mathbb{T2}\ &\leftarrow \ V\ |\ (\mathbf{\tilde\Lambda}_\mathbb{T2}\,\mathbf{\tilde\Lambda}_\mathbb{T2}) \ |\ (\mathbf{\tilde\Lambda}_\mathbb{T2}\,\mathbb{T2}) \ |\ (\lambda V:\mathbb{T2}.\, \mathbf{\tilde\Lambda}_\mathbb{T2})\ |\ (\lambda V:*.\, \mathbf{\tilde\Lambda}_\mathbb{T2})
  \end{align*}
\end{definicja}

Wyrażenia \(\lambda\) (\(\lambda\)-termy) w systemie \(\lambda 2\) to klasy abstrakcji \(\alpha\)-konwersji. Konstrukcja jest analogiczna do zaprezentowanej w Rozdziale \ref{subsec:lambda_terms_untyped}. Oczywistej modyfikacji ulega relacja \(\alpha\)-konwersji.

  \begin{definicja}(\(\alpha\)-konwersja)
    \begin{enumerate}
      \item
        \(\lambda x:\sigma.\,M =_\alpha \lambda y:\sigma.\,M\)
      \item
      \item
    \end{enumerate}
  \end{definicja}

  \subsubsection{Typowanie}
  \begin{definicja}(Kontekst)
  \end{definicja}
  Typowanie: \(\Gamma \vdash M : A\)\\
  \begin{center}
  \begin{tabular}{r c c}

    \vspace{0.5cm}
    (var) &
      \(\Gamma \vdash x:\sigma\), & jesli \(x:\sigma\in\Gamma\)\\
    \vspace{0.5cm}

    (app) &
    {\begin{prooftree}
      \Hypo{\Gamma \vdash M:\sigma \to \tau} \Hypo{ \Gamma \vdash N:\sigma}
      \Infer2[]{\Gamma \vdash MN:\tau}
    \end{prooftree}} & \\
    \vspace{0.5cm}

    (abs) &
    {\begin{prooftree}
      \Hypo{ \Gamma, x:\sigma \vdash M:\tau }
      \Infer1[]{\Gamma \vdash (\lambda\, x:\sigma.\, M):\sigma\to \tau}
    \end{prooftree}} & \\
    \vspace{0.5cm}

    (form) &
      \(\Gamma\vdash B:*\), & jeśli \(B\in\mathbb{T}2\) i  \(\mathrm{FV}(B)\subseteq \mathrm{rg}\,\Gamma\)\\
    \vspace{0.5cm}

    (\(\Pi\)e) &
    {\begin{prooftree}
      \Hypo{ \Gamma \vdash M:(\Pi \alpha:*.\,A)}
      \Hypo{ \Gamma \vdash B:* }
      \Infer2[]{\Gamma \vdash M B : A[\alpha/B]}
    \end{prooftree}} &
     \\
    \vspace{0.5cm}

    (\(\Pi\)i) &
    {\begin{prooftree}
      \Hypo{\Gamma, \alpha : * \vdash M:A } 
      \Infer1[]{\Gamma \vdash \lambda \alpha:*.\,M:\Pi \alpha:*.\,A}
    \end{prooftree}} & \\
  \end{tabular}
  \end{center}
  \subsubsection{Redukcja}
  \begin{definicja}(\(\beta\)-redukcja)
  \end{definicja}




