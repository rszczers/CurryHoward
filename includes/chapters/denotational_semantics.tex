Pojęcia \emph{funkcji} używa się na ogół mając na myśli jedno z dwóch znaczeń:
\begin{enumerate}[label=(\arabic*)]
  \setlength\itemsep{0em}
  \item funkcji jako algorytmu, którym \emph{obliczamy} wartość dla zadanego argumentu,\label{operational_sem}
  \item funkcji w rozumieniu teoriomnogościowym: jako zbioru par ar\-gu\-ment-wartość z którego wartość \emph{odczytujemy}.
\label{denotational_sem}
\end{enumerate}
Ujęcie \ref{operational_sem} nazywane jest \emph{semantyką operacyjną}. Oddaje ono dynamiczny charakter procesu obliczania wartości funkcji jako ciągu wykonywanych w czasie elementarnych operacji na zadanym argumencie. W~kontekście teorii języków programowania przez operacje elementarne możemy rozumieć wykonywanie podstawowych instrukcji procesora. W~teorii obliczalności to samo rozumielibyśmy pod pojęciem \emph{funkcji obliczalnej}, zaś algorytmiczny proces otrzymywania wartości nazwalibyśmy \emph{efektywnym}.

Ujęcie \ref{denotational_sem} odpowiada rozumieniu funkcji jako ustalonego, statycznego zbioru przyporządkowań z którego możemy odczytać wartość. Przypisanie funkcjom takiego znaczenia nazywamy semantyką \emph{denotacyjną}. Wymaga ono dostępu do pełnej informacji o funkcji. Niestety, spełnienie tego wymogu na ogół nie może być efektywnie zrealizowane ze względu na złożoność pamięciową konieczną do przeprowadzenia takiego procesu (tzw. \emph{memoizacji}).
